\begin{abstract}
Recent interest in mobile Internet has resulted in increasing
research studies about personal devices such as smartphones and
tablets. Because of their proximity to people, if used properly,
can be of tremendous value to the research community. However,
the privacy and security challenges have also increased
dramatically. In this paper we introduce Sensibility Testbed, a
testbed with security and privacy techniques that ensure the
security of user-owned devices, and the privacy of
user-generated data. The use model of Sensibility Testbed is
unique in that it (1) manages how device owners make their
devices accessible to the research community, and (2) offers
experiment resources to researchers that allow them to collect
data from remote mobile devices without rendering these devices
at risk.
					
Sensibility Testbed provides privacy mechanisms to protect
device owners' privacy by mediating data access according to
researcher's IRB policies. The testbed also uses security
measures to prevent any inadvertent or malicious bugs in
experiment code and thus protects the devices. Sensibility
Testbed allows safe research without rendering devices
vulnerable, and lowers the barriers for researchers to perform
research experiment on end users' mobile devices.

\end{abstract}