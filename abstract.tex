\begin{abstract}
%Unfortunately, the expanded use of
%these devices also has lead to increased privacy and security 
%challenges, which makes research on these devices difficult.
The research potential of personal devices, such as smartphones 
and tablets, has sparked a number of recent research initiatives. 
Because these devices are omnipresent, they could be of 
tremendous value to the research community. However, the 
expanded use of these devices has resulted in increased privacy 
and security challenges. The question remains whether this type 
of research can be carried out without causing breaches of personal 
privacy. 

In this paper we present Sensibility Testbed, an experiment  
platform for advancing the use of personal mobile devices in research. 
%security and privacy measures to ensure the 
%The use model of Sensibility Testbed is
%unique in that it (1) manages how device owners make their
%devices accessible to the research community, and (2) offers
Sensibility Testbed provides privacy protection of mobile device 
data, and maintains the security of device systems from 
potentially buggy experiment code. The experiment code in Sensibility 
Testbed runs in isolated sandboxes to 
prevent inadvertent or malicious bugs. Access 
to sensor values is mediated according to the policies 
set by the institutional review board (IRB) of the researcher's 
university or institution. Policies are enforced by a set of 
\textit{blurring layers} via the secure sandbox. As a result, 
this allows Sensibility Testbed to cater to a wider range of 
participants than prior mobile testbeds. Furthermore, 
Sensibility Testbed's platform design encourages reusing 
experiment code and infrastructure, and saves effort on 
recruiting participants and managing a deployment. 
%using a combination of blurring and rate-limiting. 
%beyond that of previous campus-only,
%incentivized testbeds, thereby adding to the diversity of the
%installed base in terms of participating devices, OS software
%versions, network operators, participant demographics, countries,
%etc.
%
%experiment measures to researchers that allow them to collect
%data from remote mobile devices without rendering these devices
%at risk. 
%					
%Sensibility Testbed provides privacy mechanisms to protect
%device owners' privacy by mediating data access according to
%researcher's IRB policies. The testbed also uses security
%measures to prevent any inadvertent or malicious bugs in
%experiment code and thus protects the devices. 

Sensibility Testbed lowers the barriers for researchers to perform
research experiment on end users' mobile devices, and 
allows safe research without rendering devices vulnerable.

\end{abstract}