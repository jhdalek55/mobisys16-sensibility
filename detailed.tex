\subsection{Example Scenario}\label{sec-detailed}
%\cappos{The detailed example needs to be more precise / detailed.  I would
%favor this using an actual example (Alice, the researcher wants to do a
%human motion study which is running on Bob's phone).  Then this section
%could tie in all of the pieces as are needed.  Note, I think this same example
%should also be used early in the paper, but at that point will not explain
%how the system causes this to happen.}
%Here we are going to put the detailed walkthrough. Putting this text here for now.
%\yanyan{may work better as a subsection of the previous section.}

From a researcher's specified IRB policies to running experiment code, 
the process goes as follows. We use the same example as in 
Section~\ref{sec-walkthrough}, where researcher Bob wants to
know the cellular service quality using Alice's device.


%\subsubsection{Researcher registers experiment and provides IRB policies}
%\label{sec-case2}
Bob's first step is to 
%access the testbed begins when he
%needs to provide a set of etailed IRB policies from his institution. 
%In order to obtain an IRB approval, a researcher first 
%To do this, Bob 
fill out an experiment registration form at the 
clearinghouse. The clearinghouse registration page shows 
a list of sensors accessible to testbed users, and 
each of their possible accuracy and access frequency limits. 
%\lois{I think I already asked this, but is it literally just the sensors that the page shows, and not the devices containing the sensors that are shown? It's a small point, but I think it has to be clarified} 
In this case, Bob specifies that 
%what data can be accessed by a research experiment, at which 
%granularity or frequency such data can be accessed, how data 
%should be securely stored, and so on. \yanyan{cite register 
%experiment website url.}
his experiment can (1) read location information
from devices at the granularity of a city, (2) read accurate
cellular signal strength and network type, as well as
%but not allow access to information about 
cell IDs, and (3) get location and
cellular network data updates every 10 minutes. 
%Bob submits an
%experiment description for these requirements, which the
%clearinghouse will codify into policies that are later enforced
%on remote mobile devices (Section~\ref{sec-ch}).
%Bob then uses this form, along with other forms downloaded 
%from the clearinghouse, to apply for IRB approval at his institution.
%
After filling out this form, Bob downloads the experiment description 
he provided, the detailed information about Sensibility Testbed and 
several relevant forms, such as 
those addressing consent, terms of participation (for device owners),  
terms of usage (for the researcher), and so on.  
Bob then uses these forms as a template to complete the IRB application 
with his institution. These forms serve as a set of reference documents 
to make it easier for researchers like Bob to 
file the necessary IRB paperwork with their institutions.

After the application is submitted, Bob's IRB may disagree with 
his initial experiment requirements. For example, Bob's IRB will not permit
his experiment to access cell IDs in cellular networks, but 
approves the other policies. 
%Bob wants to access cell 
%IDs in cellular networks, but his IRB disallows such data access. Bob then
In this case, Bob will revise the experiment registration form, refile the paperwork, 
and obtain IRB approval. Bob then submits the revised  
registration form and his IRB approval to the clearinghouse.

The clearinghouse creates a list of access policies for Bob's
experiment, according to the specified IRB policies. For the sensor
data in Table~\ref{tab:default}, the clearinghouse identifies the names of the 
corresponding blurring layer, and then parses Bob's registration 
form, extracts each data accuracy and access frequency limits as the 
input parameters to the blurring layer code (to blur an exact location 
to a city center, disallow access to cell IDs, and allow cellular network and 
location query once every 10 minutes). Finally, the clearinghouse 
assigns an experiment account to Bob. 

Once his account is activated, 
%Bob obtains his \textit{authentication keys} assigned by the 
%clearinghouse. These keys are to authenticate Bob with the 
%clearinghouse and the set of devices to which he has access.
%\path{bob.public} and \path{bob.private}. Next, 
Bob obtains his public/private key pair, and can request a number of devices from the 
clearinghouse (Section~\ref{sec-ch}). The clearinghouse then looks up
available devices (details described in technical 
report~\cite{zhuangTR15}). If the clearinghouse 
discovers that Alice's device is available, it
assigns her device to Bob's experiment account by placing Bob's
public key in Alice's sandbox. The sandbox on Alice's device obtains 
a list of command-line arguments from the clearinghouse, including all 
the blurring layer names and parameters for each layer. The pre-set blurring 
layers determine the type of data required, such as location, and Bob's 
IRB parameters customize the specific policy, such as accessing
the location information at a rate of once per 10 minutes. At this point, 
Bob can remotely access Alice's device through his experiment 
manager and can stop and start the experiment.

%the first of which must be the encasement library.
%The kernel reads in the encasement library code and uses
%the virtual namespace abstraction to execute the code with
%the exported kernel functions. The encasement library
When Bob starts his experiment, the sandbox %uses its blurring layer creation call to 
instantiates the first blurring layer according to its contract, i.e., the function 
mapping that contains the kernel's exported functions.
%the security layer instantiation call, and the remaining
%command-line arguments. 
The newly instantiated blurring layer repeats this process 
%using the 
%encasement library's
%blurring layer creation call 
to instantiate the next security layer with a potentially updated contract and 
function mapping. Eventually, Bob's program is instantiated
in a separate layer with the functions provided
through the stack of blurring layers that preceded it.
Bob's program will then be subject to all the 
policies defined in the preceding layers, or the policy stack. 
As a result,  the policies are transparently applied to Bob's experiment code. 

The mechanisms in this section are all transparent to the experimenters 
and device owners, as the implementation of policies is controlled by the 
clearinghouse on behalf of the experimenters. An experimenter is aware 
of certain policies in place, but does not need to implement or explicitly
enforce such policies. 

\begin{comment}
As already described, the clearinghouse has a default set of blurring layers 
for accuracy and access frequency levels for each sensor. 
The clearinghouse first instructs 
the sandbox on Alice's device to add Bob's policies by preloading
a set of blurring layer code. It then supplies the extracted data from 
Bob's registration form as input parameters to the blurring layers. 
%that will be instantiated on Alice's device. 
The policies are easily customizable. Details about
how policies are implemented will be introduced in Section~\ref{sec-policy}.
\end{comment}

%\textbf{Usage scenario 4: Researcher acquires device(s) and runs an experiment.}
%\label{sec-acquire-run}
%The above clearinghouse protocol ensures the enforcement of data
%access policies. Additionally, 
%To perform an experiment, Bob needs to request the use of some devices. 
%Recall that a testbed-specific key, \path{key.sensibility}, is distributed
%with the Sensibility Testbed app downloaded and installed by device
%owners (Section~\ref{sec-owner-participate}). 
%
%\yanyan{Albert thinks this is too much detail.}
%At this moment,  Bob has obtained an account with the clearinghouse.
%and is assigned a pair of public and 
%private keys, \path{key.bob-pub} and \path{key.bob-priv}, by the
%clearinghouse. 
%If the clearinghouse finds that Alice's device is available from the 
%lookup service, it
%%adds Bob's public key, \path{key.bob-pub}, to
%%the sandbox on Alice's device. This indicates that Bob is
%%authorized to use this sandbox on Alice's device, and 
%assigns Alice's device to Bob's experiment account by placing Bob's
%public key on Alice's device. It then instructs 
%the sandbox on Alice's device to add Bob's policies by preloading
%a set of blurring layer code. At this point, Bob can access Alice's 
%device through the experiment manager, just like using \texttt{ssh}.
%Bob writes his experiment 
%code in the Python-like language supported by our secure sandbox.
%The following is a snipet of code that gets location coordinates 
%from a device:
