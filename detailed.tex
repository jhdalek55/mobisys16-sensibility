\subsection{Example Scenario}\label{sec-detailed}
\cappos{The detailed example needs to be more precise / detailed.  I would
favor this using an actual example (Alice, the researcher wants to do a
human motion study which is running on Bob's phone).  Then this section
could tie in all of the pieces as are needed.  Note, I think this same example
should also be used early in the paper, but at that point will not explain
how the system causes this to happen.}
%Here we are going to put the detailed walkthrough. Putting this text here for now.
%\yanyan{may work better as a subsection of the previous section.}

From a researcher's specified IRB policies to running experiment code, 
the process goes as follows. 
The clearinghouse creates a list of access policies for a researcher's
experiment, according to the specified IRB policies. The sandbox 
on a mobile device, under the control of
this experimenter, obtains a list of command-line arguments 
from the clearinghouse, which includes all the blurring layers
and parameters for each layer (Section~\ref{sec-ch}). The pre-set blurring layers determine
the type of data required, such as accelerometer access rate, and the 
IRB parameters customize the specific policy, such as accessing
an accelerometer at a rate of 50 times/second. 
%the first of which must be the encasement library.
%The kernel reads in the encasement library code and uses
%the virtual namespace abstraction to execute the code with
%the exported kernel functions. The encasement library
The sandbox then %uses its blurring layer creation call to 
instantiates 
the first blurring layer according to its contract, i.e., the function 
mapping that contains the kernel's exported functions.
%the security layer instantiation call, and the remaining
%command-line arguments. 
The newly instantiated blurring layer repeats this process 
%using the 
%encasement library's
%blurring layer creation call 
to instantiate the next
security layer with a potentially updated contract and function
mapping. Eventually, the experimenter's program is instantiated
in a separate layer with the functions provided
through the stack of blurring layers that preceded it.
The experimenter's program will then be subject to all the 
policies defined in the preceding layers, or the policy stack.

The mechanisms in this section are all transparent to the experimenters 
and device owners, as the implementation of policies is controlled by the 
clearinghouse on behalf of the experimenters. An experimenter is aware 
of certain policies in place, but does not need to implement or explicitly
enforce such policies. 
