\section{Evaluation}\label{sec-eval}

In this section we evaluate the effectiveness of Sensibility Testbed's 
privacy mechanisms, its usability as a mobile testbed, and its 
performance.

%\todo{Q: how well the testbed protects privacy, and still allows experiments
%to function? A: \ref{sec-accurate} and \ref{sec-function}.}


\subsection{Sensibility Testbed's Privacy Policies}

\begin{table}
\scriptsize
\centering

\bgroup
\def\arraystretch{1.15}% % for table padding
\begin{tabular}{|l|l|c|c|c|}
\hline
\multirow{2}{.8cm}{\bf Project} & \multirow{2}{*}{\bf Sensor} & 
\multicolumn{3}{c|}{\bf Equivalent Sensibility Testbed policy} \\\cline{3-5}
& & {\bf Default} & {\bf Specialized} & {\bf N/A} \\\hline

EnCore~\cite{aditya2014encore}  & Bluetooth & \tickmark &   &   \\\hline

\multirow{2}{*}{\cite{chen2014sensor}\textsuperscript{*}} & Accelerometer 
& \tickmark &   &  \\ \cline{2-5}
& Camera & & \tickmark & \\ \hline

\multirow{5}{.8cm}{ProtectMyPrivacy \cite{agarwal2013protectmyprivacy}} & Device ID & & \tickmark & \\ \cline{2-5}
& WiFi & \tickmark &   &  \\ \cline{2-5}
& Bluetooth & \tickmark &   & \\ \cline{2-5}
& Address book & & \tickmark & \\ \cline{2-5}
& Location & \tickmark &   &   \\\hline
 
\multirow{4}{*}{CQue~\cite{parate2013leveraging}}  & Location & \tickmark &  & \\\cline{2-5}
& Accelerometer & \tickmark &   &  \\ \cline{2-5}
& Gyroscope & \tickmark &   &  \\ \cline{2-5}
& Bluetooth & \tickmark &   &   \\\hline

\multirow{2}{*}{MaskIt~\cite{gotz2012maskit}} & Location & \tickmark &   & \\\cline{2-5}
& Cellular & \tickmark &   &   \\\hline

\multirow{3}{*}{Jigsaw~\cite{lu2010jigsaw}} & Accelerometer 
& \tickmark &   &  \\ \cline{2-5}  
& Microphone  & & \tickmark & \\ \cline{2-5}
& Location & \tickmark &   &   \\\hline

Cach{\'e}~\cite{amini2011cache} & Location & \tickmark &   &   \\\hline

\cite{jiang2012isolating}\textsuperscript{*} & Cellular & \tickmark &   &   \\\hline

\multirow{2}{*}{TapPrints~\cite{miluzzo2012tapprints}} & Accelerometer 
& \tickmark &   &  \\ \cline{2-5}  
& Gyroscope & \tickmark &   &  \\ \hline

\multirow{4}{.8cm}{FindingMiMo \cite{shin2011findingmimo}} 
& WiFi & \tickmark &   &  \\ \cline{2-5}
& Location & \tickmark &  & \\\cline{2-5}
& Accelerometer & \tickmark &   &  \\ \cline{2-5}
& Magnetometer & \tickmark &   &  \\ \hline

ACCessory~\cite{owusu2012accessory} & Accelerometer & \tickmark &   
&  \\ \hline

TouchLogger~\cite{cai2011touchlogger} & Gyroscope & \tickmark &   
&  \\ \hline

\multirow{6}{*}{MockDroid~\cite{beresford2011mockdroid}} 
& Location & \tickmark &  & \\\cline{2-5}
& Internet\textsuperscript{\dag} & \tickmark & & \\ \cline{2-5}
& Cellular & \tickmark &   &  \\ \cline{2-5}
& Address book & & \tickmark & \\ \cline{2-5}
& Device ID & & \tickmark & \\ \cline{2-5}
& Broadcast intent & \tickmark &   &  \\ \hline

\multirow{6}{*}{Guardian \cite{zhang2015leave}} 
& Bluetooth & \tickmark &   & \\ \cline{2-5}
& Internet\textsuperscript{\dag} & \tickmark & & \\ \cline{2-5}
& Microphone  & & \tickmark & \\ \cline{2-5}
& Cellular & \tickmark &   &  \\ \cline{2-5}
& Motion sensors & \tickmark &   &  \\ \cline{2-5}
& CPU usage\textsuperscript{\ddag} & \tickmark & & \\\hline

\cite{aviv2012practicality}\textsuperscript{*} & Accelerometer & \tickmark &   
&  \\ \hline

\multirow{2}{*}{\cite{cai2012practicality}\textsuperscript{*}} & Accelerometer 
& \tickmark &  &  \\ \cline{2-5}
& Gyroscope & \tickmark & &  \\ \hline

AccelPrint~\cite{dey2014accelprint} & Accelerometer & \tickmark &   
&  \\ \hline

\multirow{8}{*}{\cite{bojinov2014mobile}\textsuperscript{*}} & Microphone  
& & \tickmark & \\ \cline{2-5}
& Accelerometer & \tickmark &   &  \\ \cline{2-5}
& Gyroscope & \tickmark & &  \\ \cline{2-5}
& Magnetometer & \tickmark &   &  \\ \cline{2-5}
& Ambient light & \tickmark &   &  \\ \cline{2-5}
& Location (GPS) & \tickmark &   &  \\ \cline{2-5}
& Touch screen & & & \xmark \\ \cline{2-5}
& Camera & & \tickmark & \\ \hline

Gyrophone~\cite{michalevsky2014gyrophone} & Gyroscope 
& \tickmark & &  \\ \hline

\cite{shokri2011quantifying}\textsuperscript{*}
& Location & \tickmark &   &  \\ \hline

\cite{polakis2015s}\textsuperscript{*}
& Location & \tickmark &   &  \\ \hline

AnonySense~\cite{kapadia2008anonysense} 
& Location & \tickmark &   &  \\ \hline

\cite{liu2015good}\textsuperscript{*} 
& Accelerometer & \tickmark &   &  \\ \hline

LP-Guardian~\cite{fawaz2014location} 
& Location & \tickmark &   &  \\ \hline

\cite{bordenabe2014optimal}\textsuperscript{*}
& Location & \tickmark &   &  \\ \hline

PrivStats~\cite{popa2011privacy}
& Location & \tickmark &   &  \\ \hline

\multirow{5}{*}{ipShield~\cite{chakraborty2014ipshield}} 
& Location (GPS) & \tickmark &   &  \\ \cline{2-5}
& Accelerometer & \tickmark &   &  \\ \cline{2-5}
& Gyroscope & \tickmark & &  \\ \cline{2-5}
& WiFi & \tickmark &   &  \\ \cline{2-5}
& Cellular & \tickmark &   & \\ \hline
 
\multirow{2}{*}{\bf Total} & \multirow{2}{*}{\bf 64} & \multirow{2}{1cm}{\bf 
54/64 (84.38\%)} & \multirow{2}{1cm}{\bf 9/64 (14.06\%)} & 
\multirow{2}{1cm}{\bf 1/64 (1.56\%)} \\ & & & & \\\hline


\multicolumn{5}{l}{\textsuperscript{*}\scriptsize These projects do not have a project name.} \\ 

\multicolumn{5}{l}{\textsuperscript{\dag}\scriptsize Internet connectivity policies
can be implemented by interposing socket calls.} \\

\multicolumn{5}{l}{\textsuperscript{\ddag}\scriptsize CPU usage can be obtained
through reading the files in \path{/proc/stat}.} \\

\end{tabular}
\egroup

\caption{\small Sensibility Testbed's support for privacy and security policies. A default 
policy is supported by Sensibility Testbed without extra effort. A specialized policy can 
be supported by moderate effort. A policy is marked as N/A if it is not possible to provide
support.}
\label{tab:policy}
%\vspace{-10pt}
\end{table}

%\todo{Q1: does ST protect privacy sufficiently?}


\textbf{Privacy and security concerns.}
In order to identify the current privacy and security concerns on mobile 
devices, we surveyed X projects in the past 5 to 6 years. These projects, listed in Table~\ref{tab:policy},
range from social network applications~\cite{aditya2014encore} to facial
recognition algorithms~\cite{chen2014sensor}. Out of these projects, 
Y of them proposed privacy protection about location information, Z of 
them had concerns about wireless network such as WiFi and Bluetooth
(connection/pairing history, MAC addresses, etc.), and M of them 
considered motion sensors such as accelerometer and gyroscope are
risky. 
%\todo{Q1.1: what privacy concerns do people have? }

\textbf{Are Sensibility Testbed's policies sufficient?}
As shown in Table~\ref{tab:policy}, xx\% of the security and privacy
issues in prior projects can be addressed in Sensibility Testbed using
default policies, and yy\% issues can be resolved by extending 
default policies (specialized policies). 

In particular, prior work shows 
that Android users' touch inputs can be revealed through a few attack 
techniques like keyloggers and fingerprinting. For example, a smartphone's accelerometer 
and gyroscope can disclose shift and rotation data when a user types 
through a software keyboard. Imprecisions in sensor calibration can 
result in a device-specific scaling and thus can be a reliable fingerprint 
of the device. User generated data thus can be informative enough 
for malware to infer the key the user enters~\cite{cai2011touchlogger, 
owusu2012accessory}, or to identify individual devices~\cite{bojinov2014mobile, 
dey2014accelprint}. Because these motion sensors are accessible 
through JavaScript in a mobile web browser, without requesting any 
permissions or notifying the device owner, these attack techniques are much 
less detectable. The chance for these techniques to succeed
depends on their sampling rate. 
%Consider an Android 
%user's average typing speed of 3 keys per second, when the sampling 
%rate goes down to once per second, the best the adversary can do is 
%just to identify 1 of these 3 keys. 
Therefore, the policies to restrict the access rates to motion sensors 
are effective if, e.g., the allowed rate is lower than the best keylogger 
would require to identify a key.

Another line of attack is on location privacy. 

%\todo{Q1.2: does ST address these concerns? A1.2: show
%  a handful of past experiments with a 90/10 rule -- 90\% of experiments
%  can be doone with virtually no mods, and the other 8\% we can write 
%  specialized policy for and 2\% it's too hard.}

\subsection{Privacy Protection and Experiment Functionality}

\todo{Q2: is ST useful for answering research questions? A:
compare the results of an algorithm with varying levels of 
data precision (Seth's algorithm?), and show a diagram suggested
by Justin.}

\yanyan{with the privacy protection in place, is the data we 
provide sufficient for experiments to function?}

%\subsection{Usability}\label{sec-usability}
%
%\yanyan{user survey on privacy: show how people feel if their 
%privacy has been protected, whether device owners feel the 
%protection is enough.}
%
%\yanyan{incentives to participants}

\subsection{Deployment Experience}\label{sec-deployment}

\todo{Q3: how is the testbed being used? A: deployment experience: 
number of people, projects, issues, hackathon.}

%\subsubsection{External Projects and Collaboration}\label{sec-external}

Sensibility Testbed has been adopted by projects such as 
Open3G~\cite{open3g} that investigates on cellular technologies 
and coverage, a vehicle data collection project~\cite{reininger2015first} 
that monitors vehicle traffic patterns. The experience from the 
project developers was positive, and they identified cases where our 
instructions for use were unclear. Despite these clarification issues, in 
the case of Open3G, the developer reported that writing the code to
get cellular information took about an hour and they have used our
testbed since 2012. We are currently in discussions with other 
researchers about integrating Sensibility Testbed into their research
projects, such as monitoring construction safety.

In 2014, we hosted a Hack-a-thon styled, one-day workshop co-located with 
the IEEE Sensors Applications Symposium (SAS) \cite{sas}. About twenty 
workshop participants with various backgrounds worked in teams 
for six hours and built four functional applications using Sensibility 
Testbed. None of the participants had any prior experience with 
this testbed, and many of them had no background in Computer
Science. With this success, we hosted a second workshop with 
SAS in 2015, and will continue in 2016. Interesting experiments 
developed by the participants include a device network monitor: 
when the battery level of a device is low, WiFi or Bluetooth that 
requires high power is turned off.

\subsection{Ease of Use}\label{sec-ease}

\todo{Q4: how easy is it to use the testbed? A: ease of use: 
compared to common measurement app (Android), 
how many lines of code can one save (\ref{sec-ease})}

\yanyan{how much time a student spent doing some
measurements (like Thomas's motion detection alg).}

\yanyan{compare to other projects,  developers can save
thousands of lines of code, and reuse the same set of 
devices.}

\subsection{Microbenchmarks}\label{sec-benchmark}

\todo{Q5: what's the overhead? A: CPU/memory overhead 
and battery (\ref{sec-benchmark}).}

\yanyan{briefly show how effective/efficient it collects data, 
and overhead compared to native.}

\subsubsection{Measurement Overhead}

\subsubsection{Battery Consumption}
