\section{Implementation}\label{sec-policy}
\cappos{This needs a bit of work.  It's not just Repy we borrow.}
%\jill{can we add more implementation details in this section about other components in addition to Repy?}
In this section, we describe how to control the precision of sensor data (Section~\ref{sec-layer}) 
and the frequency of sensor access (Section~\ref{sec-nanny}). 
It is important to acknowledge that the Sensibility Testbed 
benefitted from the design and implementation of an earlier experimental 
platform for networking and distributed system research called Seattle. 
First deployed six years ago, Seattle continues to run in a safe and 
contained manner on tens of thousands of computers around the world. It 
can be used for applications in cloud computing, peer-to-peer networking, 
mobile computing, and distributed systems. A core design principle of its 
operation is the installation of secure sandboxes on user devices. These 
sandboxes, installed automatically when a device owner downloads the app, 
limit the consumption of resources such as CPU, memory, storage space, and 
network bandwidth, allowing Seattle to run with minimal impact on system 
security and performance. Programs are only allowed to operate inside of a 
sandbox, ensuring that other files and programs on the computer are kept 
private and safe. The basic design of the Repy Sandbox described below was 
leveraged from the earlier Seattle design. \jill{what do we add?}

\begin{comment}
\subsection{Sensor Access from the Repy Sandbox}



\yanyan{this subsection is too detailed. maybe put in a TR, except the
last paragraph.}

%\lois{why not call it the Repy Sandbox as before?}

%In the use cases in Section~\ref{sec-scenario}, when Alice starts 
%the Sensibility Testbed app, the native code in the app initializes a
%Python interpreter, launches the Repy sandbox, and starts the 
%communication between the device and the clearinghouse. The 
%sandbox's API provides calls to file system, networking, 
%threading functions, and so on. Therefore, Bob's code can read 
%files, send data through the network, etc., from Alice's device. 
%As in Section~\ref{sec-repy}, the Repy sandbox does not include 
%calls to access sensors. But 
To allow a researcher's code to access sensors, such as 
GPS, WiFi, Bluetooth, accelerometer, and cellular network, the Repy  
sandbox needs access to native Android code. \lois{I think I already 
asked this, but does this only run on Android devices? If so, I think 
that needs to be mentioned.} We first implemented a set of sensor 
functions using Java native code. \lois{Again, who implemented the sensors? And are you referring to the design initially done that is now in the oast? Or the actions taken by a user now} The Repy 
sandbox then uses a Remote Procedure Call (RPC) to invoke the
corresponding Java code in Python, and returns the data 
to a sandboxed Repy program. This defines a set of sensor APIs in 
Repy's programming language, such as \path{get_location()}, 
\path{get_accelerometer()}, \path{get_wifi()}, etc. 

\begin{figure}
\begin{Verbatim}
1. \textcolor{Purple}{def} \textbf{\textcolor{NavyBlue}{get_location}}():
2.   \textcolor{BrickRed}{"""}
3.   \textcolor{BrickRed}{Get raw location data from GPS, network or passive.}
4.   \textcolor{BrickRed}{"""}
5. 
6.   \textcolor{BrickRed}{# start the locating process} 
7.   sensorlib.request_data(\textcolor{BrickRed}{'startLocating'})
8.
9.   \textcolor{BrickRed}{# try to read current location}
10.  location = sensorlib.request_data(\textcolor{BrickRed}{'readLocation'})
11.
12.  \textcolor{BrickRed}{# stop the locating process} 
13.  sensorlib.request_data(\textcolor{BrickRed}{'stopLocating'})
14.
15.  \textcolor{Purple}{if not} location:
16.    \textcolor{Purple}{raise} LocationNotFoundException    
17.  
18.  \textcolor{Purple}{return} location
\end{Verbatim}
\caption{\small Sandbox implementation of \texttt{get\_location()}. 
\label{fig-getlocation}}
\end{figure}

Figure~\ref{fig-getlocation} shows how \path{get_location()} 
is defined as a sandbox function which receives unfiltered location 
information from a mobile device. 
On line 7, \path{sensorlib.request_data()} is an RPC call 
defined in Repy, 
%\path{sensor_socket} is the socket for the Repy code to communicate with the native code, 
and \path{startLocating} is the name of the native Java method that tells the Android 
location manager to start looking up location information. Lines 10 and 13 are similar RPC 
calls that read location information from the Android location manager, and stops the location 
lookup. The call \path{get_location()} thus %is defined in the namespace of the sandbox kernel, it 
can be directly used in an experiment program, and can 
accomplish in one line of Python code that would otherwise require several dozen 
lines of Java code. Thus it makes writing an experiment program quicker and more convenient.

A full list of sensor API is documented at~\cite{sensor-api}. The set
of sensors range from battery, Bluetooth, and cellular, to location, WiFi, 
accelerometer, and so on. 
As such, the original Repy interface and the added sensor API together 
provide the complete \textit{OS level} sandbox kernel on a mobile 
device, as shown in Figure~\ref{fig-blur}.

\end{comment}

\subsection{Secure Sandbox}\label{sec-sandbox}
The Repy sandbox is the most important in the device software. This 
security-reviewed sandbox~\cite{cappos2010retaining} mitigates the 
impact of bugs in experimenter code by providing security isolation 
and performance isolation\footnote{\scriptsize 
This sandbox has been used successfully for more than six years in our 
prior work, the Seattle testbed~\cite{seattle}.}. 
%Instead of developing full-fledged Android apps, all 
%experiments in Sensibility Testbed are written in a language
%similar to Python, and run in a secure %Python-based
Researchers use a Python-like programming interface~\cite{repyv2}
to write experiment code, upload the code and execute it in the
sandbox on remote devices. The programming interface includes functions for networking, 
file system access, threading, locking, logging, and so on. To access sensors, 
the sandbox also has a set of sensor functions~\cite{sensors}. 
%Details about implementing the Repy sandbox for mobile 
%devices are described in Section~\ref{sec-repy-ext}.
%
%However, the current Repy sandbox does not include calls to access sensors. 
%%To obtain the sensor data, we need to extend the sandbox. 
%The extended Repy sandbox that allows sensor access  
%will be described in Section~\ref{sec-repy-ext}.
Another important feature of the Repy sandbox allows us to change the 
behavior of its programming interface, and control the 
data gathered from the device to adapt to any IRB proscribed limits. 
%For  an experiment
%that involves GPS location, a privacy policy might restrict the
%level of data granularity available to the experiment. For example, it can
%obfuscate GPS location such that it only identifies the center
%of the city that the device is located in, rather than the exact
%location. Using the same technique, 
For example, the sandbox can anonymize the IP address of a device, constrain  
the frequency of access to GPS locations, and disable
access to cameras. 
%Such privacy protection is a contribution of Sensibility Testbed, 
%which does not exist in any prior work. 
The details of policy implementation are presented in 
Section~\ref{sec-layer} and \ref{sec-nanny}.

\textbf{Security and performance isolation.}
The sandbox is the execution environment 
of all experiment code. It isolates experimenters' programs from one 
another, and prevents any of the code from harming the device 
through its strong isolation techniques. More importantly, the sandbox
provides a flexible system call interposition technique where system call behavior 
can be modified. This allows us to define and implement
different data access policies. 

Repy has a small, self-contained kernel as its trusted 
computing base (TCB). %unprivileged libraries and experiment code. 
Every system call in the TCB is strictly 
sanitized to preserve consistent behavior across different OSes, 
and to avoid uncommon corner cases that can be exploited. 
Furthermore, since the TCB is small (about 8,000 lines of code), it is 
less likely to have a kernel exploit than other more complex kernels. 
As a result, any vulnerability in an experiment 
%can at most cause compromise in the unprivileged portion of the sandbox, 
cannot escape the sandbox and perform malicious actions to 
the device system. The Repy sandbox exposes its system calls for 
accessing resources on a device through a Python-like programming 
interface~\cite{repyv2} isolating experiment code from the kernel. 
%another way of saying this:
%Experiment code for accessing resources is contained by the 
%Repy sandbox, using its Python-like programming language interface.
This same sandbox design has been successfully running experiments as part of a network research 
testbed for more than six years, without significant operation faults or security breaches~\cite{seattle}. \yanyan{Justin thinks this statement is weak.} \lois{I don't know if this answers his concerns since the meaning is essentially the same}
%=======
%through a Python-like programming interface~\cite{repyv2}. 
%%to experiment code for accessing resources on a device. 
%%another way of saying this:
%%Experiment code for accessing resources is contained by the 
%%Repy sandbox, using its Python-like programming language interface.
%This same sandbox design has been successfully running experiments 
%as part of a network research testbed~\cite{seattle} for more than six 
%years, with no report of operational faults or security breaches. 
%>>>>>>> yyzhuang/master

In addition to security isolation,
%The sandbox exposes a programming 
%interface to experimenter programs to access 
%resources on the device. 
the sandbox provides performance isolation by
%by using OS hooks to monitor and control the amount of 
%CPU and memory used by a sandboxed program. To restrict 
%other resources, such as network and disk I/O, the sandbox 
interposition on system calls~\cite{garfinkel2003traps} that 
use resources, such as network and disk I/O, and preventing 
or delaying the execution of these calls if they exceed 
their configured quota. 
%As a result, this sandbox limits 
%what a sandboxed program can do. For example, 
This isolation means the sandbox can set access limits. For example, 
reading from and writing to the file system can
only occur in a per-experiment directory, sending and receiving
data via the network interface cannot exceed a configured rate, and 
CPU, memory and battery consumption cannot exceed a set level.
Therefore, the sandbox isolates the experiment program from 
the rest of the device. Due to this isolation, different researchers 
can run experiments on different sandboxes on the same device, 
without any interference between the experiments.


%\subsubsection{Resource Manager} The second part of the device software, called a resource manager, 
%establishes a trust relationship between a device and a researcher. It 
%manages a researcher's access to the sandboxes on a device, and
%controls the start and stop of the sandboxes on the device. The 
%resource manager controls which researchers can access the 
%sandbox on any given device, and communicates with the clearinghouse
%or experiment manager. If multiple researchers have access to 
%the same device, the resource manager splits the resources in the 
%sandbox among the researchers, and partitions the one sandbox 
%into several smaller ones. In order for the clearinghouse or a 
%researcher's experiment manager to discover the sandbox on a 
%device, the resource manager also contacts a lookup service that 
%announces the existence of the available devices. An example is 
%given in Section~\ref{sec-case1}.

\begin{comment}
\subsection{Policy Enforcement}

\subsubsection{Reducing Data Precision}\label{sec-layer}

%\yanyan{TODO: replace all security layers with blurring layers. fix the line wrapping.}
%The Sensibility Testbed uses an extended version of the Repy sandbox, whose 
%security mechanism is described in our prior work by Cappos, {\it et 
%al}~\cite{cappos2010retaining}. This section briefly explains this sandboxing 
%technique, and how it is used in Sensibility Testbed.

%In Sensibility Testbed, we construct a researcher's IRB policies as a set of 
%isolated and contained reference monitors called blurring layers. They each
%implement one IRB policy, and can 
%be stacked together to form a policy stack. 

Policies are implemented by each blurring layer via a \textit{virtual 
namespace} that provides function mapping to substitute raw 
data access with restricted data access. 
%and the boundary between two blurring layers is monitored by an 
%\textit{encasement library} that verifies interface semantics at runtime. 
To ensure the runtime behavior of each function mapping (for example, 
a function mapping of \path{get_location()} still returns location data), 
every blurring layer uses a \textit{contract} to verify the interface 
semantics between the blurring layers.

\textbf{Virtual namespace.}
The virtual namespace of each blurring layer executes the code in descendant 
layers with the corresponding layer's function mapping. By our convention, 
the virtual namespace of a blurring layer does not contain functions from the 
sandbox kernel or the namespace of its parent layer, unless explicitly 
specified by the mapping, ensuring that if an API call is not specifically
allowed, it is blocked.
For example, if a layer \path{foo} with 
functions \path{get_battery()}, \path{get_accelerometer()}, and 
\path{restricted_get_accelerometer()} were to instantiate a descendant 
layer \path{bar} with a function mapping 

\begin{Verbatim}
\{
\textcolor{BrickRed}{'get_battery'}: get_battery, 
\textcolor{BrickRed}{'get_accelerometer'}: restricted_get_accelerometer
\}
\end{Verbatim}
then the module \path{bar} would have access
to \path{foo.get_battery} via the name \path{get_battery}, and to
\path{foo.restricted_get_accelerometer} via the name 
\path{get_accelerometer}. That is, the function on the left of the colon \texttt{:}
\textit{replaces} the function on the right via the virtual namespace. 
In this case, layer \path{bar} would not be able to access 
\path{foo.get_accelerometer}.

\textbf{Contract.}
%
%The virtual namespace is useful for loading
%code dynamically, but does not provide adequate security for
%use as an isolation boundary. The encasement library in the 
%Repy sandbox provides isolation between virtual namespaces. 
%Security layers do not share objects or functions.
%The encasement library copies all objects that are passed
%between security layers.
%
Each function that can be called by descendant blurring
layers needs to have its arguments, return value, and 
other runtime behavior verified. This concept is similar to system call filtering
mechanisms~\cite{acharya2000mapbox, fraser2000hardening} 
that mediate access to a sensitive function interface. In our case, 
the functions are the calls to smartphone sensors that 
can potentially reveal a device owner's private information. Our
verification process uses a contract. For each function that has a 
mapping, the contract lists the number 
and types of its arguments, the exceptions that can be raised 
by the function, and the return type of the function\footnote{\scriptsize 
Since Python is a dynamically-typed language, it is useful to type check 
a function's arguments, exceptions, and return values.}.


A contract is represented as a Python dictionary in Repy. As an example, if 
the blurring layer \path{foo} wanted to create a contract that would map
\path{get_battery()} and \path{restricted_get_accelerometer()} into 
a new namespace, the contract would be: 

\begin{Verbatim}
\{\textcolor{BrickRed}{'get_battery'}: \{
  \textcolor{BrickRed}{'type'}: \textcolor{BrickRed}{'func'},
  \textcolor{BrickRed}{'args'}: \textcolor{Purple}{None}, 
  \textcolor{BrickRed}{'exceptions'}: (\textcolor{Purple}{BatteryNotFoundError}), 
  \textcolor{BrickRed}{'return'}: dict,
  \textcolor{BrickRed}{'target'}: get_battery
  \}, 
\textcolor{BrickRed}{'get_accelerometer'}: \{
  \textcolor{BrickRed}{'type'}: \textcolor{BrickRed}{'func'},
  \textcolor{BrickRed}{'args'}: str, 
  \textcolor{BrickRed}{'exceptions'}: (\textcolor{Purple}{ValueError}), 
  \textcolor{BrickRed}{'return'}: list,
  \textcolor{BrickRed}{'target'}: restricted_get_accelerometer
  \}
\}
\end{Verbatim} 

Note that the symbols in the contract come from \path{foo}'s 
namespace. Thus the target for the \path{get_battery} in the 
contract is the \path{foo.get_battery} function. Similarly, the 
target for the \path{get_accelerometer} in the contract is the 
\path{foo.restricted_get_accelerometer}.

Whenever an experiment calls a function, such as \path{get_battery}
or \path{get_accelerometer}, a verification process must perform 
type-checking using the contract. If the verification process 
detects a semantic violation, \yanyan{add an example} the entire experiment program is 
terminated. This guarantees that to a sandboxed program, 
the implemented policies do not change a function's semantics.
%In addition to type checking, the verification function copies 
%arguments and return values of mutable types to prevent 
%time-of-check-to-time-of-use bugs. Since mutable types are 
%copied, the caller cannot cause a race condition by modifying objects.


Each blurring layer provides a contract to instantiate the 
next blurring layer by substituting a version of the function that 
enforces a given policy. All descendant layers (layers loaded 
after this layer) will have access to the version of the function 
that enforces this new policy. The final blurring layer starts the 
experiment program with the appropriate set of functions. 
%
%\subsubsection{Blurring Layer Instantiation}
%\smallskip
%From start to finish, the entire process proceeds as follows. 
%The clearinghouse creates a list of access policies for a researcher's
%experiment, according to the specified IRB policies. The sandbox on a mobile device, under the control of
%this experimenter, obtains a list of command-line arguments 
%from the clearinghouse, which includes all the blurring layers
%and parameters for each layer. The pre-set blurring layer determine
%the type of policy required, such as accelerometer access rate, and the 
%IRB parameters customize the specific policy, such as accessing
%an accelerometer at a rate of 50 times/second. 
%%the first of which must be the encasement library.
%%The kernel reads in the encasement library code and uses
%%the virtual namespace abstraction to execute the code with
%%the exported kernel functions. The encasement library
%The sandbox then %uses its blurring layer creation call to 
%instantiates 
%the first blurring layer according to its contract, i.e., the function 
%mapping that contains the kernel's exported functions.
%%the security layer instantiation call, and the remaining
%%command-line arguments. 
%The newly instantiated blurring layer repeats this process 
%%using the 
%%encasement library's
%%blurring layer creation call 
%to instantiate the next
%security layer with a potentially updated contract and function
%mapping. Eventually, the experimenter's program is instantiated
%in a separate layer with the functions provided
%through the stack of blurring layers that preceded it.
%The experimenter's program will then be subject to all the 
%policies defined in the preceding layers, or the policy stack.
%
Therefore, the experimenter's program is ensured to use all the 
new policies. An example of blurring layer implementation is given in 
%Section~\ref{sec-precision-example}.  
our technical report~\cite{zhuangTR15}. 
\end{comment}

\begin{comment}
\subsubsection{An Example of Policy Implementation}
\label{sec-precision-example}

%\subsubsection{Reducing Data Precision}

Below is an example of the implementation of a location blurring layer. 
When researcher Bob requests a device, the clearinghouse recognizes that all the location data
he collects must be substituted with a blurred location that only identifies the nearest city.
%rather than the exact location of the device. 
Therefore, the following blurring layer is
automatically loaded, along with Bob's experiment code, to map 
the \path{get_location()} call in Figure~\ref{fig-getlocation} to a version 
that implements such a policy. \yanyan{load code or load policy text? how
to translate policy text to code?}In the following, we name this
blurring layer \path{blur_to_city}, and assume it is the first descendant
of the sandbox kernel.

\begin{Verbatim}
1. \textcolor{Purple}{def} \textbf{\textcolor{NavyBlue}{get_city_location}}():
2.   \textcolor{BrickRed}{"""}
3.   \textcolor{BrickRed}{This function replaces the exact coordinates of} 
4.   \textcolor{BrickRed}{the Android device with the coordinates for the } 
5.   \textcolor{BrickRed}{geographic center of the nearest city.}
6.   \textcolor{BrickRed}{"""}
7.
8.   location = get_location()
9.
10.  closest_city = find_closest_city(location[\textcolor{BrickRed}{"latitude"}],
11.    location[\textcolor{BrickRed}{"longitude"}])
12.
13.  location[\textcolor{BrickRed}{"latitude"}] = closest_city[\textcolor{BrickRed}{"latitude"}]
14.  location[\textcolor{BrickRed}{"longitude"}] = closest_city[\textcolor{BrickRed}{"longitude"}]
15.
16.  \textcolor{Purple}{return} location
17.
18. \textcolor{BrickRed}{# Substitute get_location with get_city_location.}
19. \textbf{CHILD_CONTEXT_DEF[\textcolor{BrickRed}{"get_location"}] = \{}
20.    \textbf{\textcolor{BrickRed}{"type"}: \textcolor{BrickRed}{"func"},}
21.    \textbf{\textcolor{BrickRed}{"args"}: \textcolor{Purple}{None},}
22.    \textbf{\textcolor{BrickRed}{"return"}: \textcolor{Purple}{dict},}
23.    \textbf{\textcolor{BrickRed}{"exceptions"}: \textcolor{BrickRed}{"any"},}
24.    \textbf{\textcolor{BrickRed}{"target"}: get_city_location,}
25. \textbf{\}}
26.
27. \textbf{secure_dispatch_module()}
\end{Verbatim}


Line 1 -- 16 defines a new function named \path{get_city_location()} which returns 
the nearest city to the device. On line 10 -- 11, the function \path{find_closest_city()}
uses a quadtree-like algorithm~\cite{gruteser2003anonymous} that subdivides 
the area around the device until the area contains at least a city.
This algorithm can be applied similarly if a policy requires location 
precision at the state/province or country level.
%in \path{blur_to_city}'s namespace that is outside the sandbox kernel. 
Lines 19 -- 25 defines a contract for \path{get_city_location()}. 
%and stores it in a global dictionary \path{CHILD_CONTEXT_DEF}. 
In this contract, 
%the target for the sandbox kernel function 
%\path{get_location()} (line 19) is \path{get_city_location()} (line 24).
%This means that 
\path{get_city_location()} will substitute \path{get_location()} defined 
in the Repy sandbox kernel.
%On Line 19, \path{CHILD_CONTEXT_DEF} indicates that this data structure is to replace the
%Repy library function \path{get_location()}. 
Line 20 - 23 define the function's type, arguments,
return values, and any potential exceptions for runtime verification. 
%Line 24 indicates that Repy library function
%\path{get_location()} will be replaced by \path{get_city_location()} defined in line 1 - 16, once
%this blurring layer is in effect, which means the blurring layer is enforced via running along a
%sandboxed program. Finally, in order for this blurring layer to take effect, we need to call
Finally, \path{secure_dispatch_module()} on line 27 instantiates this
\path{blur_to_city} layer, when running along with the sandbox kernel. 
\yanyan{or does it instantiate experiment code?}As a result, 
whenever Bob's experiment code calls \path{get_location()}, 
the \path{blur_to_city} layer replaces it with \path{get_city_location()}. 

Similarly, because Bob's IRB policy disallows access to cell ID, another blurring layer 
can substitute the \path{get_cellID()} call in the sandbox kernel with a function that
returns \path{None}. This layer can be instantiated by \path{blur_to_city}
as its descendant, or by any other layer. An arbitrary number of layers can 
be stacked together to implement an experimenter's policies.
%
All experiment code uses the same function call defined by the sandbox, 
and only the blurring layers change the function's behavior, thus 
the policy enforcement is transparent to all the experimenters who run 
sandboxed programs in Sensibility Testbed.


\subsubsection{Restricting Data Access Frequency}\label{sec-nanny}

%A function call 
%request can be met only if the frequency is no greater than the cap. 
%\textbf{Data precision and data access frequency.}
%\yanyan{I feel this may better be placed in implementation section.}
Reducing data precision partially protects a device owner's privacy. However, frequent data 
access can be another channel to snoop on personal information.
If sensor queries are made on a sufficiently frequent basis, they can be used
to track an individual's activity. The goal of reducing sensor access frequency 
is to prevent an accumulation of identifiable information, such as 
location~\cite{gruteser2003anonymous}, wireless network identifiers, and even 
accelerometer data. (Note that specific
details as to how frequently a certain sensor can be accessed without risking such an 
invasion is out of the scope of this work). 
Furthermore, the battery power of a device can 
be drained faster if sensors are accessed
with unnecessary frequency. Therefore, restricting the data access rate 
mandates that experiments collect data only as often as is necessary. 
To achieve this, the Repy sandbox provides a second mechanism
called \path{nanny} that mediates and restricts access frequency to sensors. 

\path{nanny} treats all sensors as \textit{resources}, and the function calls to 
access the sensors as the \textit{usage} of resources. 
%\path{nanny}'s control 
%mechanism for a resource is to limit the \textit{rate} of its usage. For example, 
When an 
%Utilization is controlled over one or more periods, where
experiment program's use of a resource is above a given threshold, 
\path{nanny} pauses the 
program for as long as required to bound it, on average, below the
threshold. Therefore, if an experiment program attempts to 
use a resource at a rate faster than is allowed by a policy, the function 
call is blocked until sufficient time has passed to average out the overall access rate. 
To monitor and control the usage of resources, \path{nanny} keeps a 
table of resource assignments that tracks and updates requests and releases. 
Once resource caps are set, an experiment program can never call a 
function to access a sensor more frequently than the cap. 

An example implementation is given in technical report~\cite{zhuangTR15}.
\end{comment}

\begin{comment}
\subsubsection{An Example of Policy Implementation}\label{sec-rate-example}

Restricting access rate can also be implemented as a 
blurring layer, in a similar way as reducing data precision.
Recall that in Section~\ref{sec-component}, Bob's IRB policy requires that
his experiment can get location updates every 10 minutes (600 seconds). 
Therefore, the following blurring layer is automatically loaded along with 
Bob's experiment code (likely as a descendant of another layer). In 
the following, we name this blurring layer \path{restrict_location}.

\begin{Verbatim}
1.  \textcolor{BrickRed}{# allow get_location call once per 600 seconds}
2.  \textbf{resourcesdict = \{\textcolor{BrickRed}{'get_location'}: 1.0/600\}} 
3.
4.  \textbf{nanny.start_resource_nanny(resourcesdict)}
5.
6.  \textcolor{Purple}{def} \textbf{\textcolor{NavyBlue}{restricted_get_location}}():
7.    \textbf{nanny.tattle_quantity(\textcolor{BrickRed}{'get_location'}, 1)}
8.    location_data = get_location()
9.    return location_data
10.
11. CHILD_CONTEXT_DEF[\textcolor{BrickRed}{"get_location"}] = \{
12.   \textcolor{BrickRed}{"type"}: \textcolor{BrickRed}{"func"},
13.   \textcolor{BrickRed}{"args"}: \textcolor{Purple}{None},
14.   \textcolor{BrickRed}{"return"}: \textcolor{Purple}{dict},
15.   \textcolor{BrickRed}{"exceptions"}: \textcolor{BrickRed}{"any"},
16.   \textcolor{BrickRed}{"target"}:  restricted_get_location,
17. \}
18. 
19. secure_dispatch_module()
\end{Verbatim}

Line 2 above defines a resource assignment table, \path{re- sourcesdict}, to track and update 
the function call to \path{get_location()}. It restricts the frequency to 
call \path{get_location()} to once per 600 seconds (note that 
600 seconds is filled in by Bob, and parsed by the clearinghouse as a blurring layer 
parameter to \path{restrict_location}, as in Section~\ref{sec-ch}). Line 4 initializes \path{nanny} 
with the resource assignment table, and lines 6 - 9 define a 
function called \path{restricted_get_location()} that restricts location updates. 
The \path{tattle_quantity()} call on line 7 charges \textit{one usage} of \path{get_location()} 
call. It tells \path{nanny} that in \path{restricted_get_location()}, 
\path{get_location()} (line 8) will be called exactly once. The
next time this experiment program 
%calls \path{get_location()} again, the program 
will be paused for 600 seconds before it can call \path{get_location()} again,
as defined by \path{resourcesdict}.

The contract on line 11 - 17 defines that the target
for \path{get_location()} is \path{restricted_get_location()}. If 
\path{restrict_location} is the first descendant of the sandbox kernel, then
\path{restricted_get_location()} will replace the \path{get_location()}
in the sandbox virtual namespace. If there is a preceding blurring layer 
before \path{restrict_location} that already updated its 
function mapping for \path{get_location()} (like \path{blur_to_city}  
in Section~\ref{sec-precision-example}), then
\path{restricted_get_location()} will replace the \path{get_location()} call
with the policy implemented in this preceding blurring layer. In this case, 
the experiment code will get location updates once per 10 minutes, at
the city level. 
%This is because layers \path{blur_to_city} and 
%\path{restrict_location} are both used in the policy stack.

\bigskip
%\subsubsection{Blurring Layer Instantiation}
%\bigskip
\end{comment}



\subsection{Resource Manager and Device Manager}
The \sysname resource and device managers are implemented as Python 
programs that are set up and configured as the \sysname Android app 
is installed. As with the sandbox, both components are based on our 
prior work in Seattle Testbed, but are specifically customized to 
the Android OS. For instance, starting the app at boot time of the 
device is implemented by registering a \texttt{BroadcastIntent} handler 
for \path{ACTION_BOOT_COMPLETED}. Also, permanent storage needs 
special treatment, as the media on which a \sysname install lives 
might be unmounted and ejected from the device at any time.
The resource manager furthermore includes code to dynamically 
traverse NAT gateways and firewalls (both which are commonly found 
on Internet connections over WiFi), so that it remains contactable 
even if the device does not currently possess a public IP address.


\subsection{Clearinghouse}
\albert{I don't know if any of this is interesting at all.}

\sysname's clearinghouse is hosted on a web server operated by the 
authors. \albert{...trying to indicate that ``you, dear reader'' needn't worry about that aspect...}

It is implemented on top of the Django web framework. 
Django uses Python as its runtime (as does our sandbox), so it is 
straightforward for the clearinghouse to reuse the same libraries for 
cryptography, node announcement and lookup, and sandbox control as the 
other parts of the system.



\subsection{Experiment Manager}

This component is required to run on experimenter's machines so that 
they can interface with the sandboxes they can control. The experiment 
manager, including assorted Repy libraries to support the development of 
experiments, is available in packaged form \cite{demo-kit}, and runs 
on all operating systems and platforms supported by Python 2. 
To accommodate for the reality of today's networks, the experiment 
manager supports contacting resource managers on devices 
inside WiFi networks (or behind NAT gateways and firewalls in general).
