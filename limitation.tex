\section{Practical Challenges and Limitations}\label{sec-limitation}

Below we discuss the limitations of \sysname 
when designing experiments to run on the testbed, 
and consider various challenges we met while building the system.

\subsection{Limitations}\label{subsec-limitations}
In \sysname, experiment code runs in a sandbox, and 
can access sensors only through privacy-preserving blur layers. 
In contrast to a native app (in our case, an Android app), this starkly curtails 
the experiment's access to the mobile device. For example, \sysname 
does not provide a way to interact with the device owner 
through unsolicited on-screen messages or notifications that a native 
app could send\footnote{\scriptsize Interactivity is still possible. 
An experiment may serve a web 
page on the local device through the web browser, and interact with 
the device owner as a web app.}. 

Similarly, \sysname limits access to other parts 
of the native API. For example, there is no way for an experiment to 
signal that it wants to handle an \texttt{Intent}~\cite{intent}, e.g. 
act as an MP3 player to open a file of this type, or handle phone calls, 
etc.~\cite{intent-dev}.
%\albert{%Playing back audio files, handling calls and text messages etc. 
%%are examples of things implemented via Intents in Android, see 
%%\url{https://en.wikipedia.org/wiki/Intent\_\%28Android\%29} and 
%%\url{https://developer.android.com/reference/android/content/Intent.html}. 
%Do you think we should explain the concept of Intents more?}
The operations above 
could well be added to the sandbox API -- there is no technical 
obstacle for this. Rather, it was our fundamental security design 
choice for \sysname to not allow these.


\subsection{Challenges}\label{subsec-challenges}

We discuss the existing challenges of building \sysname. We hope to 
resolve these challenges as we gain more practical experience with 
\sysname.
We approach questions of use, usefulness, technical design, and
general architecture.


\textbf{Use.}~\yanyan{This doesn't sound like a challenge?}
From the perspective of a researcher designing an experiment to 
run on \sysname, there is a certain learning phase 
to go through in order to familiarize with the test and deployment 
tools, the sandbox API, operational aspects of a distributed system, 
and so on. Since the sandbox uses a different programming language 
than Android usually does, and the patterns to access sensor values 
are different as well, no previous experience with Android is required 
to program for \sysname. Indeed, from our experience in 
educational contexts, an experimenter who brings along any programming 
knowledge will quickly learn how to use the sandbox, while specific 
Android experience does not necessarily provide an advantage.
% Also, the sandbox API is small, and stays away from unconventional 
% or OS-specific patterns, so it is easy to learn even for the
% uninitiated.


\textbf{Usefulness.}~
\sysname's current feature set makes it most 
useful as a passive, distributed, large-scale sensor network. 
It does not map well to interactive crowd sensing, crowd sourcing, 
and (geo) tagging tasks, for studies looking at personal data stored 
on the phone (such as performing research using the device owner's 
phone book),
% Cf. https://en.wikipedia.org/wiki/Six_degrees_of_separation
or introspection into the Android OS (which, e.g., PhoneLab provides). 
In short, we expect other approaches to remain valid and interesting 
besides \sysname, but hope that its scale and distribution 
are attractive properties nevertheless.

%\albert{Could add questions regarding scale, too:
%``Is S.T. a meaningful way to scale out smartphone research?'',
%``Shall we scale it out at all, or let Google / the mobile operators 
%do it''.}

Another interesting aspect we hope to understand better as 
we open \sysname to the public is what motivates device 
owners to altruistically ``donate'' resources to researchers. 
Since \sysname currently provides no native (GUI-based, 
notification) way for experiments to interface with the device 
owner, we hypothesize that altruism plays a role as an incentive. 
We note that there are other non-interactive sensor projects that 
keep records of 
resource donations (like a high-score table), let contributors form 
teams, and so on~\cite{OpenWLANMap}.
% With due thanks to Karl Karpfen for pointing out the project!
This rewards contributors with some form of acknowledgment and 
publicity. Other projects appear to have utilized pure altruism to 
great effect. For example, the authors of~\cite{wang2011untold} 
claim hundreds of installs from the Android app 
store, suggesting that there exists a sizable population of device 
owners that will install an app that gives them neither an obvious, 
direct benefit from interactivity, nor a form of publicity like 
the high-score table.


\textbf{Technical challenges.}~
\sysname is designed to minimize the privacy repercussions 
of smartphone research by limiting an experiment's access to sensor values.
For this, we provide blurring layers that transform the sensor 
values, round them or otherwise limit their accuracy, replace them by 
random data, limit the access rate, and so on (Section~\ref{sec-policy}). 
In Section~\ref{sec-eval} we show that blurring is a meaningful strategy to 
counterbalance a researcher's desire for sensor access and a device 
owner's desire for privacy, and that our implementation of blurring 
can express complex interrelations between sensor values and accuracies. 
As noted in Section~\ref{sec-policy-design}, we anticipate that 
the default policies will need adaptation over time, i.e. restrict 
access even more severely.
%However, this does not guarantee that the existing blurring layers and 
%configurations will always be able to address adequately all possible 
%privacy concerns. 
Besides, bugs in the sandbox or blur code might 
expose the device owner's privacy, thus requiring updates of the 
platform code.
%To meet these objections, \sysname includes a software 
%updater so that we can fix any security issues and 
%improve the coverage provided by the blurring layers.

%\albert{Another possible point is that interface-compatble, 
%transpartent, drop-in blur functions might confuse more than 
%they are useful, because there is no clear indication of the 
%accuracy and rate that you get (cf. Leon's student's presentation 
%on 2015-11-15). This is more criticism of drop-in blur rather than 
%blur as such.}

Letting the device owner choose and parametrize blurring layers 
also poses a usability 
challenge we currently try to resolve.
%Section~\ref{subsec:informed-consent} 
%discusses the stack of policies (device owner's, our default ones, 
%the researcher's IRB's) that allows access to sensors at certain 
%accuracy levels and rates.
% \albert{We now claim that we do support a Sensorium-like GUI to control sensor access, so I cut the next sentence.}
%For the device owner, intervention is currently limited to a binary 
%choice, opt-in or opt-out. This provides a streamlined yet little 
%expressive way for device owners to decide on participation in an 
%experiment. 
In our prior work~\cite{sensorium, rafetseder2013sensorium}, 
we evaluated a graphical user interface to control sensor access in 
a fine-grained manner, but found it difficult to 
balance expressiveness (when and how accurately which 
sensor or group of sensors may be accessed) and usability (i.e., 
the number and description of user interface widgets that enable 
the device owner to express their desired policy).
Expressiveness is great for concerned device owners 
that want to make informed decisions. 
However, we need to understand better whether our existing interface 
satisfies the device owners and helps them express their desired 
policy correctly.



\textbf{Architectural challenges.}~
Further questions remain as regards the capability of solving privacy 
issues using sensor blur as such. For example, there likely exist 
accuracy-vs-time trade-offs so that a malicious researcher might compensate 
a low sensor read-out rate by instead running their experiment for a 
very long time, and exploiting temporal autocorrelations in the measured 
values, e.g. stemming from diurnal human activity.
% \albert{I guess we argue against "signing away rights" all the time, so I'll skip the next proposed challenge.}
%Also, deviow's sign away their rights with app installation anyway, 
%don't they? Why bother with blur then?

Lastly, 

Are IRBs the right way to solve the problem? E.g., 
rogue researchers that deliberately obfuscate what they will 
do with sensor readings, so as to breach privacy. E.g., 
ignorant IRBs that greenlight everything and put the burden on 
our default IRB (????).

