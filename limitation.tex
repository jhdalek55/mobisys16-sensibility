\section{Practical Challenges and Limitations}\label{sec-limitation}

Below we discuss potential deficiencies of Sensibility Testbed 
that should be kept in mind when designing experiments to run on it, 
and consider various challenges we met while building the system.

\subsection{Limitations}\label{subsec-limitations}
In Sensibility Testbed, experimenter code runs inside a sandbox, and 
can access sensors only through privacy-preserving blur layers. 
In contrast to a native Android research app, this starkly curtails 
the experiment's access to the Android device. For example, Sensibility 
Testbed does not provide a way to interact with the device owner 
through unsolicited on-screen messages or notifications that a native 
app could send\footnote{Interactivity still is possible if the user 
chooses to make the first move: An experiment may serve a web page on 
the local device through the web browser, and interact with the user 
as a ``web app''.}. 

On similar terms, Sensibility Testbed limits access to other parts 
of the Android API. For example, there is no way for an experiment to 
signal that it wants to handle \texttt{Intent}s of any sort, e.g. 
act as an MP3 player whenever the device owner wants to open a file 
of this type.

It should be noted that the operations in the above examples 
could well be added to the sandbox API -- there is no technical 
obstacle for this. Rather, it was our fundamental security design 
choice for Sensibility Testbed to not allow these.


\subsection{Challenges}\label{subsec-challenges}

\begin{itemize}
  \item It's not given that a testbed that runs non-interactive 
experiments is the right answer to ``How can we scale out 
smartphone research'', ``Shall we scale it out at all'', etc.
  \item Researchers could require interactivity, privacy-invasive 
sensor access, and other things that don't map well to S.T.
  \item Are IRBs the right way to solve the problem? E.g., 
rogue researchers that deliberately obfuscate what they will 
do with sensor readings, so as to breach privacy. E.g., 
ignorant IRBs that greenlight everything and put the burden on 
our default IRB (????).
  \item Are blur layers the right way to solve the problem? E.g., 
collect enough data to make use of temporal autocorrelation; secure enough.
  \item Granularity of user interaction is on opt-in / opt-out 
level. No way currently for users to add own blur, cf. Sensorium 
which does have this.
  \item You have to learn the platform first. Deviates from Android programming, e.g. "sensor calls".
\end{itemize}

