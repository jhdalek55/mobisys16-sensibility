\section{Policy Implementation}

\subsection{Policy Stack}
In Sensibility Testbed, privacy policies are implemented as blurring layers, with each layer acting as a 
reference monitor in the sandbox~\cite{ref} to enforce an access 
control policy over each sensor. Using the sandboxing 
technique in our prior work~\cite{cappos2010retaining}, we are able to
interject code to control the behavior of sandbox functions, or 
system calls. A sensor access policy can (1) reduce 
the precision of the raw sensor data returned from a device, such
as returning the location of a nearest city rather than the device's exact location, and (2) restrict 
the frequency of access to a sensor, such as the polling rate of a gyroscope or
an accelerometer, to avoid password interference~\cite{michalevsky2014gyrophone}.

Each policy is a blurring layer, which defines one of the above
two categories of policies. Different sets of policies can be customized 
according to the regulations set by IRB, through loading 
individual blurring layers in order, as a policy stack. In each stack, 
a lower layer is the ancestor of a higher layer. Every layer inherits 
the policy defined by its ancestor layers with the exception of the lowest layer,
%Each blurring layer is untrusted by its ancestor layers, 
%but is trusted by its descendant layers. \yanyan{maybe cut this.}
which is the 
Sensibility Testbed's sandbox kernel. The experiment program runs at the top 
of the policy stack, thereby inheriting all the policies defined by the
lower layers, as shown in Figure~\ref{fig-blur}. 
Each policy stack acts as a set of filters for different sensors, through 
which a call must pass before a sandboxed program can
access the sensor data. 

We next introduce how the two categories of policies are implemented in Repy.

\subsection{Reducing Data Precision}\label{sec-layer}

%\yanyan{TODO: replace all security layers with blurring layers. fix the line wrapping.}
%The Sensibility Testbed uses an extended version of the Repy sandbox, whose 
%security mechanism is described in our prior work by Cappos, {\it et 
%al}~\cite{cappos2010retaining}. This section briefly explains this sandboxing 
%technique, and how it is used in Sensibility Testbed.

%In Sensibility Testbed, we construct a researcher's IRB policies as a set of 
%isolated and contained reference monitors called blurring layers. They each
%implement one IRB policy, and can 
%be stacked together to form a policy stack. 

Policies are implemented by each blurring layer via a \textit{virtual 
namespace} that provides function mapping to substitute raw 
data access with restricted data access. 
%and the boundary between two blurring layers is monitored by an 
%\textit{encasement library} that verifies interface semantics at runtime. 
To ensure the runtime behavior of each function mapping (for example, 
a function mapping of \path{get_location()} still returns location data), 
every blurring layer uses a \textit{contract} to verify the interface 
semantics between the blurring layers.

%\subsubsection{Virtual Namespace and Contract}

\textbf{Virtual namespace.}
The virtual namespace of each blurring layer executes the code in descendant 
layers with the corresponding layer's function mapping. 
%By our convention, 
%the virtual namespace of a blurring layer does not contain functions from the 
%sandbox kernel or the namespace of its parent layer, unless explicitly 
%specified by the mapping. \yanyan{why is this statement needed?}
For example, if a layer \path{foo} with 
functions \path{get_battery()}, \path{get_accelerometer()}, and 
\path{restricted_get_accelerometer()} were to instantiate a descendant 
layer \path{bar} with a function mapping 

\begin{Verbatim}
\{
\textcolor{BrickRed}{'get_battery'}: get_battery, 
\textcolor{BrickRed}{'get_accelerometer'}: restricted_get_accelerometer
\}
\end{Verbatim}
then the module \path{bar} would have access
to \path{foo.get_battery} via the name \path{get_battery}, and to
\path{foo.restricted_get_accelerometer} via the name 
\path{get_accelerometer}. That is, the function on the left of the colon \texttt{:}
\textit{replaces} the function on the right via the virtual namespace. 
In this case, layer \path{bar} would not be able to access 
\path{foo.get_accelerometer}.

\textbf{Contract.}
%
%The virtual namespace is useful for loading
%code dynamically, but does not provide adequate security for
%use as an isolation boundary. The encasement library in the 
%Repy sandbox provides isolation between virtual namespaces. 
%Security layers do not share objects or functions.
%The encasement library copies all objects that are passed
%between security layers.
%
Each function that can be called by descendant blurring
layers needs to have its arguments, return value, and 
other runtime behavior verified. This concept is similar to system call filtering
mechanisms~\cite{acharya2000mapbox, fraser2000hardening} 
that mediate access to a sensitive function interface. In our case, 
the functions are the calls to smartphone sensors that 
can potentially reveal a device owner's private information. Our
verification process uses a contract. For each function that has a 
mapping, the contract lists the number 
and types of its arguments, the exceptions that can be raised 
by the function, and the return type of the function\footnote{\scriptsize 
Since Python is a dynamically-typed language, it is useful to type check 
a function's arguments, exceptions, and return values.}.

A contract is represented as a Python dictionary in Repy. As an example, if 
the blurring layer \path{foo} wanted to create a contract that would map
\path{get_battery()} and \path{restricted_get_accelerometer()} into 
a new namespace, the contract would be: 

\begin{Verbatim}
\{\textcolor{BrickRed}{'get_battery'}: \{
  \textcolor{BrickRed}{'type'}: \textcolor{BrickRed}{'func'},
  \textcolor{BrickRed}{'args'}: \textcolor{Purple}{None}, 
  \textcolor{BrickRed}{'exceptions'}: (\textcolor{Purple}{BatteryNotFoundError}), 
  \textcolor{BrickRed}{'return'}: dict,
  \textcolor{BrickRed}{'target'}: get_battery
  \}, 
\textcolor{BrickRed}{'get_accelerometer'}: \{
  \textcolor{BrickRed}{'type'}: \textcolor{BrickRed}{'func'},
  \textcolor{BrickRed}{'args'}: str, 
  \textcolor{BrickRed}{'exceptions'}: (\textcolor{Purple}{ValueError}), 
  \textcolor{BrickRed}{'return'}: list,
  \textcolor{BrickRed}{'target'}: restricted_get_accelerometer
  \}
\}
\end{Verbatim} 

Note that the symbols in the contract come from \path{foo}'s 
namespace. Thus the target for the \path{get_battery} in the 
contract is the \path{foo.get_battery} function. Similarly, the 
target for the \path{get_accelerometer} in the contract is the 
\path{foo.restricted_get_accelerometer}.

Whenever an experiment calls a function, such as \path{get_battery}
or \path{get_accelerometer}, a verification process must perform 
type-checking using the contract. If the verification process 
detects a semantic violation, \yanyan{add an example} the entire experiment program is 
terminated. This guarantees that to a sandboxed program, 
the implemented policies do not change a function's semantics.
%In addition to type checking, the verification function copies 
%arguments and return values of mutable types to prevent 
%time-of-check-to-time-of-use bugs. Since mutable types are 
%copied, the caller cannot cause a race condition by modifying objects.


Each blurring layer provides a contract to instantiate the 
next blurring layer by substituting a version of the function that 
enforces a given policy. All descendant layers (layers loaded 
after this layer) will have access to the version of the function 
that enforces this new policy. The final blurring layer starts the 
experiment program with the appropriate set of functions. 
%
%\subsubsection{Blurring Layer Instantiation}
%\smallskip
%From start to finish, the entire process proceeds as follows. 
%The clearinghouse creates a list of access policies for a researcher's
%experiment, according to the specified IRB policies. The sandbox on a mobile device, under the control of
%this experimenter, obtains a list of command-line arguments 
%from the clearinghouse, which includes all the blurring layers
%and parameters for each layer. The pre-set blurring layer determine
%the type of policy required, such as accelerometer access rate, and the 
%IRB parameters customize the specific policy, such as accessing
%an accelerometer at a rate of 50 times/second. 
%%the first of which must be the encasement library.
%%The kernel reads in the encasement library code and uses
%%the virtual namespace abstraction to execute the code with
%%the exported kernel functions. The encasement library
%The sandbox then %uses its blurring layer creation call to 
%instantiates 
%the first blurring layer according to its contract, i.e., the function 
%mapping that contains the kernel's exported functions.
%%the security layer instantiation call, and the remaining
%%command-line arguments. 
%The newly instantiated blurring layer repeats this process 
%%using the 
%%encasement library's
%%blurring layer creation call 
%to instantiate the next
%security layer with a potentially updated contract and function
%mapping. Eventually, the experimenter's program is instantiated
%in a separate layer with the functions provided
%through the stack of blurring layers that preceded it.
%The experimenter's program will then be subject to all the 
%policies defined in the preceding layers, or the policy stack.
%
Therefore, the experimenter's program is ensured to use all the 
new policies. An example of blurring layer implementation is given in 
Section~\ref{sec-precision-example}.  


\subsubsection{An Example of Policy Implementation}
\label{sec-precision-example}

%\subsubsection{Reducing Data Precision}

Below is an example of the implementation of a location blurring layer. 
When researcher Bob requests a device, the clearinghouse recognizes that all the location data
he collects must be substituted with a blurred location that only identifies the nearest city.
%rather than the exact location of the device. 
Therefore, the following blurring layer is
automatically loaded, along with Bob's experiment code, to map 
the \path{get_location()} call in Figure~\ref{fig-getlocation} to a version 
that implements such a policy. \yanyan{load code or load policy text? how
to translate policy text to code?}In the following, we name this
blurring layer \path{blur_to_city}, and assume it is the first descendant
of the sandbox kernel.

\begin{Verbatim}
1. \textcolor{Purple}{def} \textbf{\textcolor{NavyBlue}{get_city_location}}():
2.   \textcolor{BrickRed}{"""}
3.   \textcolor{BrickRed}{This function replaces the exact coordinates of} 
4.   \textcolor{BrickRed}{the Android device with the coordinates for the } 
5.   \textcolor{BrickRed}{geographic center of the nearest city.}
6.   \textcolor{BrickRed}{"""}
7.
8.   location = get_location()
9.
10.  closest_city = find_closest_city(location[\textcolor{BrickRed}{"latitude"}],
11.    location[\textcolor{BrickRed}{"longitude"}])
12.
13.  location[\textcolor{BrickRed}{"latitude"}] = closest_city[\textcolor{BrickRed}{"latitude"}]
14.  location[\textcolor{BrickRed}{"longitude"}] = closest_city[\textcolor{BrickRed}{"longitude"}]
15.
16.  \textcolor{Purple}{return} location
17.
18. \textcolor{BrickRed}{# Substitute get_location with get_city_location.}
19. \textbf{CHILD_CONTEXT_DEF[\textcolor{BrickRed}{"get_location"}] = \{}
20.    \textbf{\textcolor{BrickRed}{"type"}: \textcolor{BrickRed}{"func"},}
21.    \textbf{\textcolor{BrickRed}{"args"}: \textcolor{Purple}{None},}
22.    \textbf{\textcolor{BrickRed}{"return"}: \textcolor{Purple}{dict},}
23.    \textbf{\textcolor{BrickRed}{"exceptions"}: \textcolor{BrickRed}{"any"},}
24.    \textbf{\textcolor{BrickRed}{"target"}: get_city_location,}
25. \textbf{\}}
26.
27. \textbf{secure_dispatch_module()}
\end{Verbatim}


Line 1 -- 16 defines a new function named \path{get_city_location()} which returns 
the nearest city to the device. On line 10 -- 11, the function \path{find_closest_city()}
uses a quadtree-like algorithm~\cite{gruteser2003anonymous} that subdivides 
the area around the device until the area contains at least a city.
This algorithm can be applied similarly if a policy requires location 
precision at the state/province or country level.
%in \path{blur_to_city}'s namespace that is outside the sandbox kernel. 
Lines 19 -- 25 defines a contract for \path{get_city_location()}. 
%and stores it in a global dictionary \path{CHILD_CONTEXT_DEF}. 
In this contract, 
%the target for the sandbox kernel function 
%\path{get_location()} (line 19) is \path{get_city_location()} (line 24).
%This means that 
\path{get_city_location()} will substitute \path{get_location()} defined 
in the Repy sandbox kernel.
%On Line 19, \path{CHILD_CONTEXT_DEF} indicates that this data structure is to replace the
%Repy library function \path{get_location()}. 
Line 20 - 23 define the function's type, arguments,
return values, and any potential exceptions for runtime verification. 
%Line 24 indicates that Repy library function
%\path{get_location()} will be replaced by \path{get_city_location()} defined in line 1 - 16, once
%this blurring layer is in effect, which means the blurring layer is enforced via running along a
%sandboxed program. Finally, in order for this blurring layer to take effect, we need to call
Finally, \path{secure_dispatch_module()} on line 27 instantiates this
\path{blur_to_city} layer, when running along with the sandbox kernel. 
\yanyan{or does it instantiate experiment code?}As a result, 
whenever Bob's experiment code calls \path{get_location()}, 
the \path{blur_to_city} layer replaces it with \path{get_city_location()}. 

Similarly, because Bob's IRB policy disallows access to cell ID, another blurring layer 
can substitute the \path{get_cellID()} call in the sandbox kernel with a function that
returns \path{None}. This layer can be instantiated by \path{blur_to_city}
as its descendant, or by any other layer. An arbitrary number of layers can 
be stacked together to implement an experimenter's policies.
%
All experiment code uses the same function call defined by the sandbox, 
and only the blurring layers change the function's behavior, thus 
the policy enforcement is transparent to all the experimenters who run 
sandboxed programs in Sensibility Testbed.




\subsection{Restricting Data Access Frequency}\label{sec-nanny}

\subsubsection{\texttt{nanny}}

Reducing data precision partially protects a device owner's privacy. However, frequent data 
access can be another channel to snoop on personal information.
If sensor queries are made on a sufficiently frequent basis, they can be used
to track an individual's activity. The goal of reducing sensor access frequency 
is to prevent an accumulation of identifiable information, such as 
location~\cite{gruteser2003anonymous}, wireless network identifiers, and even 
accelerometer data. (Note that specific
details as to how frequently a certain sensor can be accessed without risking such an 
invasion is out of the scope of this work). 
Furthermore, the battery power of a device can 
be drained faster if sensors are accessed
with unnecessary frequency. Therefore, restricting the data access rate 
mandates that experiments collect data only as often as is necessary. 
To achieve this, the Repy sandbox provides a second mechanism
called \path{nanny} that mediates and restricts access frequency to sensors. 

\path{nanny} treats all sensors as \textit{resources}, and the function calls to 
access the sensors as the \textit{usage} of resources. 
%\path{nanny}'s control 
%mechanism for a resource is to limit the \textit{rate} of its usage. For example, 
When an 
%Utilization is controlled over one or more periods, where
experiment program's use of a resource is above a given threshold, 
\path{nanny} pauses the 
program for as long as required to bound it, on average, below the
threshold. Therefore, if an experiment program attempts to 
use a resource at a rate faster than is allowed by a policy, the function 
call is blocked until sufficient time has passed to average out the overall access rate. 
To monitor and control the usage of resources, \path{nanny} keeps a 
table of resource assignments that tracks and updates requests and releases. 
Once resource caps are set, an experiment program can never call a 
function to access a sensor more frequently than the cap. 

%A function call 
%request can be met only if the frequency is no greater than the cap. 

\subsubsection{An Example of Policy Implementation}\label{sec-rate-example}

Restricting access rate can also be implemented as a 
blurring layer, in a similar way as reducing data precision.
Recall that in the Alice/Bob use scenario (Testbed Operation 
in~\cite{sensibility-submission}), Bob's IRB policy requires that
his experiment can get location updates every 10 minutes (600 seconds). 
Therefore, the following blurring layer is automatically loaded along with 
Bob's experiment code (likely as a descendant of another layer). In 
the following, we name this blurring layer \path{restrict_location}.

\begin{Verbatim}
1.  \textcolor{BrickRed}{# allow get_location call once per 600 seconds}
2.  \textbf{resourcesdict = \{\textcolor{BrickRed}{'get_location'}: 1.0/600\}} 
3.
4.  \textbf{nanny.start_resource_nanny(resourcesdict)}
5.
6.  \textcolor{Purple}{def} \textbf{\textcolor{NavyBlue}{restricted_get_location}}():
7.    \textbf{nanny.tattle_quantity(\textcolor{BrickRed}{'get_location'}, 1)}
8.    location_data = get_location()
9.    return location_data
10.
11. CHILD_CONTEXT_DEF[\textcolor{BrickRed}{"get_location"}] = \{
12.   \textcolor{BrickRed}{"type"}: \textcolor{BrickRed}{"func"},
13.   \textcolor{BrickRed}{"args"}: \textcolor{Purple}{None},
14.   \textcolor{BrickRed}{"return"}: \textcolor{Purple}{dict},
15.   \textcolor{BrickRed}{"exceptions"}: \textcolor{BrickRed}{"any"},
16.   \textcolor{BrickRed}{"target"}:  restricted_get_location,
17. \}
18. 
19. secure_dispatch_module()
\end{Verbatim}

Line 2 above defines a resource assignment table, \path{re- sourcesdict}, to track and update 
the function call to \path{get_location()}. It restricts the frequency to 
call \path{get_location()} to once per 600 seconds (note that 
600 seconds is filled in by Bob, and parsed by the clearinghouse as a blurring layer 
parameter to \path{restrict_location}). Line 4 initializes \path{nanny} 
with the resource assignment table, and lines 6 - 9 define a 
function called \path{restricted_get_location()} that restricts location updates. 
The \path{tattle_quantity()} call on line 7 charges \textit{one usage} of \path{get_location()} 
call. It tells \path{nanny} that in \path{restricted_get_location()}, 
\path{get_location()} (line 8) will be called exactly once. The
next time this experiment program 
%calls \path{get_location()} again, the program 
will be paused for 600 seconds before it can call \path{get_location()} again,
as defined by \path{resourcesdict}.

The contract on line 11 - 17 defines that the target
for \path{get_location()} is \path{restricted_get_location()}. If 
\path{restrict_location} is the first descendant of the sandbox kernel, then
\path{restricted_get_location()} will replace the \path{get_location()}
in the sandbox virtual namespace. If there is a preceding blurring layer 
before \path{restrict_location} that already updated its 
function mapping for \path{get_location()} (like \path{blur_to_city}  
in Section~\ref{sec-precision-example}), then
\path{restricted_get_location()} will replace the \path{get_location()} call
with the policy implemented in this preceding blurring layer. In this case, 
the experiment code will get location updates once per 10 minutes, at
the city level. 
%This is because layers \path{blur_to_city} and 
%\path{restrict_location} are both used in the policy stack.

