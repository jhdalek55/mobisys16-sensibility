\section{Motivation}\label{sec-motivation}

\textbf{Privacy and security of mobile devices.}
Recent privacy breaches and security break-ins of mobile systems 
have resulted in higher concerns about using mobile devices like 
smartphones and tablets~\cite{breach}. Apps can post tweets to a 
user's Twitter account without asking for permission~\cite{tweet}. 
Compromised apps can even let criminals break into individual's 
bank account~\cite{starbucks}. As a result, many device owners 
are aware that running apps on their smartphones can raise privacy 
and security risks. 

However, data from the enormous number of smartphones, 
if used properly, can be of tremendous value to the research 
community, with little or no dedicated infrastructure and maintenance 
cost. For example, accelerometers on end-user devices can 
detect vibrations within the frequency and intensity range of 
seismic waves, and can assist distributed earthquake 
detection~\cite{faulkner2011next}. Cameras, GPS, WiFi, and 
cellular triangulation can be employed in distributed networks 
of sensors for traffic monitoring and accident 
prevention~\cite{mohan2008nericell, thiagarajan2009vtrack}. 
The challenge to providing data access to the research community
depends on strong protection against privacy and security breaches.

\textbf{Restricting accessible data.}
Despite the risks of using sensors, 
studies have indicated that we can access and use sensor data 
without compromising device owners' privacy or service functioning.
A recent research study has shown that more than half of the 
surveyed individuals had no problem in supplying imprecise 
sensor data from their personal devices to protect their 
privacy~\cite{fawaz2014location}. Experience also has shown 
that imprecise information is sufficient for a large class of 
services. For example, the US Federal Coomunications 
Commission requires that instead of the exact location, the 
emergency rescue and response teams are able to estimate the 911 
wireless emergency caller's position with an accuracy of 125~m 
(RMS). 

Based on these fact, restricting 
the amount of data accessible, such as reducing the precision or 
access frequency, can be a good privacy protection mechanism we can 
provide for end users. In this work, we coin the term \textit{data blurring}
as our privacy protection mechanism. This mechanism is used in 
conjunction with a researcher's IRB policies, where each data access
policy is codified as a blurring layer, and different policies are
customized by loading individual blurring layers in order.