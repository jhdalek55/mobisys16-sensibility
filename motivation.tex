\section{Motivation and Background}\label{sec-motivation}

In this section, we present some background information that 
served as the building blocks of the  design and deployment of 
Sensibility Testbed. First, we offer the motivations behind its 
development: the desire to balance the benefits of such research 
with the very real privacy risks that could potentially occur if a 
device owner allows access to his/her device. Next, we look at 
one solution to the problem: the collecting of proximate versions 
of data through the use of blurring layers. Third, we talk about 
the traditional role of IRB in research involving human subjects 
and how Sensibility Testbed can ensure these protections extend 
to data accessed remotely. Lastly, we look at Sensibility Testbed's 
default policies, which automatically closes off access to 
sensors which have the highest risk for violating the privacy of 
device owners.


\textbf{Motivation: overcoming the risks of accessing sensor data.}
Having access to data from the enormous number of smartphones 
in use today could be tremendously valuable to the research community. 
The sheer numbers of these devices, coupled with the fact that research 
on these devices do not incur high maintenance costs, makes their 
potential as information gathering instruments almost limitless. 

%Due to recent privacy breaches and security break-ins to mobile systems, 
%device security and personal privacy are genuinely at risk when a person 
%uses a smartphone or tablet \cite{breach}. 
%%Apps can post tweets to a 
%%user's Twitter account without asking for permission~\cite{tweet}. 
%A calculator app might send the user's location to an advertisement 
%server~\cite{calc}. Sensor data from the accelerometer 
%or gyroscope can be sufficient to infer the locations of touch-screen 
%taps, and thus infer a user's password~\cite{cai2011touchlogger}.
%Compromised apps can even let criminals break into an individual's 
%bank account~\cite{starbucks}. 
%%As a result, device owners 
%%are aware that running apps on their smartphones can raise privacy 
%%and security risks. 
%A major reason for the prevelant privacy breaches is that on many mobile 
%systems, such as Android, 
%only a sub-set of sensors like GPS and bluetooth are considered risky, 
%and their access is mediated~\cite{android-sec}. Other sensors 
%such as accelerometer, gyroscope, etc., 
%%are considered to be innocuous, 
%require no permission to access. Furthermore, %research shows that 
%device owners are often oblivious to the implications of granting access to
%a particular type of sensor or resource~\cite{felt2012android}. It is 
%therefore challenging to conduct research on end-user devices
%in compliance with ethical standards~\cite{zevenbergen2013ethical}.

%However, having access to data from the enormous number of smartphones 
%in use today could be tremendously valuable to the research 
%community. As these devices belong to ordinary people, conducting
%research on these devices does not incur high maintenance costs. 
%Accelerometers on end-user devices could detect vibrations within 
%the frequency and intensity range of seismic waves, and assist 
%distributed earthquake detection~\cite{faulkner2011next}. GPS, 
%WiFi, and cellular triangulation can be employed in distributed 
%networks of sensors for traffic monitoring and accident 
%prevention~\cite{mohan2008nericell, thiagarajan2009vtrack}. 
%For the research community, accessing this data depends on its 
%ability to provide strong protection to device owners from privacy 
%and security breaches. In this work, we try to address two issues. 

However, for the research community, accessing this data depends on its 
ability to provide strong protection to device owners from privacy and 
security breaches. In seeking solutions to this problem, we started with a 
few key ideas. The first is that device owners themselves may not fully 
understand the implications of granting access to a particular type of 
sensor or resource~\cite{felt2012android}. It is therefore challenging to 
conduct research on end-user devices in compliance with ethical standards, 
without some type of organized approach to enforcing those 
standards~\cite{zevenbergen2013ethical}. In addition, a major reason for 
the prevalent privacy breaches on many mobile systems, such as Android and iOS, 
is that only a sub-set of sensors like GPS and bluetooth are considered risky, 	
and therefore have their access mediated~\cite{android-sec}. Other sensors 
such as accelerometer, gyroscope, etc., require no permission to access, 
thus leaving them open to attack. Therefore, with Sensibility Testbed, 
both the inclusion of IRB restrictions (discussed later in this section), and 
the need to restrict or limit access to a wider range of sensors were both 
incorporated into the design.	

\textbf{Restricted data access and privacy protection.}
%Despite these risks of using sensors, 
Another concept important to our design is that generalized data that 
does not directly violate the privacy of the device owner can still be 
valuable. Studies have indicated that 
sensor data can be accessed without compromising device 
owners' privacy or sacrificing service functioning.
A recent research study shows that more than half of the 
surveyed individuals had no problem in supplying imprecise 
sensor data from their personal devices~\cite{fawaz2014location}. 
Most participants could accommodate some inevitable loss of application 
functionality, as long as their privacy was protected. Those surveyed
applications ranged from location-based search (e.g., Yelp), social 
network apps, to gaming and weather forecasting apps. 
Researchers thus have proposed 
substituting mocked~\cite{beresford2011mockdroid} or 
anonymized~\cite{zhou2011taming} data in place of real data. 
For example, in location-based services such as maps, 
restaurant guides, and bus schedules, end users can still use the 
service even if a device only provides a discretized 
location~\cite{amini2011cache, krumm2007inference}. Therefore, 
though the accuracy of the data is reduced, 
the imprecise information is sufficient for a large class of services. 
%The US Federal Communications Commission requires 
%emergency rescue and 
%response teams to be able to estimate a 911 wireless emergency 
%caller's position with an accuracy of 125~m~\cite{gruteser2003anonymous, 
%reed1998overview}. \yanyan{too many examples?}

Based on these facts, we decided that restricting 
the amount of data accessible, such as reducing the precision or 
access frequency, offered an effective privacy protection mechanism to 
provide to end users. In this work, we coin the term \textit{data blurring}
as our privacy protection mechanism. This concept, explained in detail in
Section~\ref{sec-policy}, automatically limits the amount of detailed 
information the device sensors will relay to the person running the 
experiment.
%where each data access
%policy is codified as a blurring layer, and different policies are
%customized by loading individual blurring layers in order.


\textbf{IRB policies: guiding ethical behavior.}
%On the other hand, research institutions have also designed a 
%protocol based on the \textit{institutional review board (IRB)}, 
%to assess the ethics of a researcher's project, and review its methods. 
Institutional Review Board serve as the ethical watch dogs for college, 
universities, government agencies, and other research institutions. 
It is the job of these boards, also known as an independent ethics committee 
(IEC), ethical review board (ERB), or research ethics board (REB), 
to approve, monitor, and review research involving human 
subjects~\cite{irb}. These groups require all researchers working under 
their aegis to submit the protocols of their studies with an aim to 
protecting not only the physical and mental well-being of subjects, 
but also to protect any information about these individuals generated 
over the course of the study. Since privacy protection of collected data 
become a bit more difficult when dealing with remote subjects, another 
issue we sought to address with Sensibility Testbed is to relieve the 
individual researcher of the need to manually enforce the restrictions 
set by his or her institution's IRB. Sensibility Testbed automates, and 
therefore ensures adoption of these policies. Although many current network 
testbeds require that researchers obtain IRB approval before conducting
an experiment on the testbed, these platforms do not provide a guarantee 
for IRB policy compliance~\cite{nandugudi2013phonelab, nikravesh2015mobilyzer}.
%In the case of PhoneLab, 
%it requires experimenters to obtain IRB approval. However, 
%it leaves it up to the experimenters to comply with their IRB policies in their 
%experiments~\cite{nandugudi2013phonelab, nikravesh2015mobilyzer}. 
%Similarly, Mobilyzer~\cite{nikravesh2015mobilyzer} provides a 
%measurement library that can be included in Android apps. 
%and requires explicit user consent. 
Therefore, there is no guarantee that an 
experiment will be compliant with a researcher's IRB policies.
%promotes fully informed 
%consent and voluntary participation by prospective subjects. 

Sensibility Testbed takes any researcher's IRB policies, codifies them, and 
uses the aforementioned blurring layers to restrict sensor access on an 
end-user's device to an institution's set access levels, making compliance 
automatic. 
%IRB plays a central role in defining the policies
%appropriate for research at individual institutions. Experimenters
%first obtain an IRB approval at their institution. Then with these IRB
%policies, Sensibility Testbed, as an intermediate, codifies the data access 
%regulations and enforces them at the end-user mobile devices. 
%This is achieved through restricting data access  via
%a set of blurring layers. Each layer implements an IRB policy by substituting 
%approximate data in place of explicit, raw sensor data to the experiment code. Different layers 
Each blurring layer implements an IRB policy by substituting approximate 
data in place of explicit, raw sensor data to the experiment code, and different 
layers together can be customized to cater to various institution's 
policies and regulations. As a result, experiments do not collect more 
data than needed to provide their functionalities.

%Our goal is to facilitate the enforcement of 
%IRB policies on behalf of researchers, and that experiments 
%do not collect more data than needed to provide their functionalities.
%This will also relieve researchers from the tedious work of 
%recruiting participants and enforcing IRB policies.

%In the domain of IRB, Alice and Bob are the participating subject, and 
%a researcher who conducts a research study on the subject, respectively.
%
%
%\textbf{Sensibility Testbed's default policies.}

\begin{table}
\scriptsize
\centering

\bgroup
\def\arraystretch{1.15}% % for table padding
\begin{tabular}{|l|c|c|c|}
\hline
\multirow{2}{*}{\bf Sensor} & 
\multicolumn{3}{c|}{\bf Default policy} \\\cline{2-4}
& {\bf LR} & {\bf MR} & {\bf HR} \\\hline

Battery (plug-in type, level, technology, etc.) & \tickmark &  & \\ \hline
Bluetooth (local name, scan mode, etc.) & & \tickmark & \\ \hline

\multirow{2}{5.5cm}{Cellular network (cell ID, area code, country code, 
operator name, etc.)} & & \multirow{2}{*}{\tickmark} & \\ 
& & & \\ \hline

Location (latitude, longitude, altitude, speed, etc.) & & \tickmark & \\ \hline
Settings (screen brightness, ringer volume, etc.) & & \tickmark & \\ \hline

\multirow{2}{5.5cm}{Motion sensors (accelerometer, 
gyroscope, magnetometer, orientation , etc.)} & & \multirow{2}{*}{\tickmark} & \\ 
& & & \\ \hline

\multirow{2}{5.5cm}{WiFi network (information about the 
currently active access point, and WiFi scan result)} & & \multirow{2}{*}{\tickmark} & \\ 
& & & \\ \hline 

%Start/stop activities & & & \xmark \\ \hline 
%Running applications & & & \xmark \\ \hline 
Camera (take pictures, record videos) & & & \xmark \\ \hline 
Intent (scan barcode, search, etc.) & & & \xmark \\ \hline 
Address book & & & \xmark \\ \hline 
Microphone (voice record) & & & \xmark \\ \hline 
SMS (send/receive messages, delete messages) & & & \xmark \\ \hline 

\end{tabular}
\egroup

\caption{\small Sensibility Testbed's default policies for sensors. LR/MR/HR
stands for low/moderate/high risk, respectively. Access is only allowed to sensors that have low to 
moderate risks (marked by \tickmark). Sensors that are highly risky are 
disabled by default (marked by \xmark).}
\label{tab:default}
%\vspace{-10pt}
\end{table}

\textbf{Sensibility Testbed's default policies.} %\label{sec-irb-policies}
The last unique concept Sensibility Testbed leveraged to ensure 
protection of end-user security and privacy protection is that all sensors 
are not alike. As mentioned earlier, failure to 
recognize the vulnerability of certain sensors was a key reason for privacy 
breaches. In designing Sensibility Testbed, default policies were set as 
to what types of sensors could be accessed. Even if an IRB happened 
to approve such a policy, there are certain sensors that the testbed's
own IRB designates as off-limits due to the high risk associated with 
potential breaches. 
%and for which access can be pre-approved with the
%researcher's local IRB. 
Only those sensors listed on our project 
wiki page~\cite{sensor-api} are accessible to a researcher. 
A summary of these sensors is listed in Table~\ref{tab:default}, 
with each one categorized as low, moderate or high 
privacy risk. The list of sensors that Sensibility Testbed provides are all of moderate 
to low privacy risks (marked by \tickmark), and the testbed further provides policy enforcement
(Section~\ref{sec-policy}) to protect all the sensor data. Sensors 
such as cameras and microphones that are deemed sensitive are not 
exposed to experiment code by default (marked by \xmark). Such 
classification is motivated by the Android system, where 
permissions are categorized into different protection levels~\cite{level}:
\textit{normal} permissions are automatically granted to the apps, 
\textit{dangerous} permissions are given based upon the 
user's consent, and so on. In our case, 
%we divide sensors into different risk levels, as shown 
%in Table~\ref{tab:default}. 
%Sensors with low to moderate risk are 
%allowed and protected by IRB policies. Sensors of high risk are 
%disabled by default. 
we divide sensors into different risk levels by the consequences and 
difficulties of a potential attack. If a microphone is controlled by 
a malicious party, it can be used to intelligently choose data of a 
higher value (e.g., credit card number, password) to record~\cite{zhang2015leave}. On the other 
hand, in order to infer a credit card number or password typed on a 
smartphone using motion sensors, the attack requires the installation of 
a sophisticated algorithm on the device that constantly learns about  
the patterns of data generated by accelerometer or gyroscope. In contrast,
using battery information alone is not sufficient to create a fingerprint 
for each device. Different information and mutiple occurrences need to
be pieced together to extract this data~\cite{battery-priv}. Therefore, 
compared to motion sensors, a microphone is considered a higher risk, 
and a battery is a significantly lower risk.

Although high-risk sensors are disabled, if such access  is critical to the 
study, access can be requested using a different IRB procedure. 
In this case, the research project has to go through the Sensibility 
Testbed's IRB, in addition to the researcher's IRB. 
\yanyan{if we think this is ok, then we provide specially
designed interface and policy?} \lois{following up on Yanyan's comment--If the Testbed's IRB says this expanded access is permissable, are the device owner's notified and can they opt out of this study? Otherwise, that would be a direct violation of the privacy protection you claim to give them}
\lois{I did not touch these last two paragraphs because I still don't know about  the opt-out policy for individuals if this permission is given}
%Depending on the experiment description provided by the 
%researcher, the fields marked with a (*) are the ones that will be blurred.
%
%

As a result, Sensibility Testbed does not
provide unfettered access to all sensors. 
%Access to sensors of
%higher risk, e.g., the policies that request restricted sensor data, 
%or at higher frequencies than our default policies, 
%needs to go through the Sensibility Testbed's IRB,
%in addition to the researcher's IRB. 
The default policies serve as a common denominator to all 
researchers' IRB policies. In most cases, we expect
that researchers need only go through their local IRB to get
the sensor access they need for their experiment. 
