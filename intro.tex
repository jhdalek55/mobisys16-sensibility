\section{Introduction}

Today's end-user mobile devices, such as smartphones and
tablets, have become indispensable gadgets in people's everyday
life. These devices have become the dominant way that 
people interact with each other and with the physical world. Because 
these devices are omnipresent, they could be of tremendous value to 
the research community. Devices in a home network generate 
unique traffic that can help providers improve the services to at-home 
users~\cite{sundaresan2011broadband}.
%Therefore, they have created increasing research opportunities. 
Mobile devices also provide a platform for researchers to
investigate how new applications can provide better performance.
For example, to improve the quality of apps, developers must 
understand how the apps perform in diverse external or internal
conditions~\cite{ravindranath2012appinsight}. Some platform
APIs even change their behavior depending on the battery level
of the device~\cite{battery}.
%The research potential of personal devices has sparked a number 
%of recent research initiatives.
As a result, there has been significant interests in the network
community to study mobile devices
(e.g., Mobilyzer~\cite{nikravesh2015mobilyzer}), and in the
systems community to deploy new services and test research
prototypes (e.g., Phonelab~\cite{phonelab, nandugudi2013phonelab}).  
					
However, there are two challenges to conducting research
on mobile devices. First, the privacy and security threats
have increased dramatically over the years. The use of 
smartphones and tablets has introduced a new class of threats, 
as code on smartphones and tablets mainly accesses
\textit{sensors} for rich functionality and user experience\footnote{\scriptsize In 
this work, we broadly define sensors as the hardware components 
that can record phenomena about the physical world, such as the 
WiFi/cellular network, GPS location, movement acceleration, etc.}. 
%Although these sensors provide convenience to device users, they 
These sensors could expose sensitive information about the device 
owner, e.g., a smartphone's GPS locations,
WiFi connections, or Bluetooth pairing history can be highly
personal. In particular, some seemingly benign apps can provide 
companies or government agency access to ordinary 
people's daily activities~\cite{AngryBirds}. Furthermore, 
a malicious party could potentially bypass a
device's security protections and gain access to user privileges. 
%We not only need to  protect the privacy
%of device owners so that the code cannot eavesdrop on phone
%conversations or infer passwords, but also ensure that 
%experiment code can not inadvertently damage  
%%or maliciously hack into 
%the device.

To lower potential security and privacy risks and to make the 
participant recruitment easier, many current testbeds choose to 
select from an internal group, such as students and 
colleagues~\cite{hao2013isleep, wang2012no, 
wang2013sensing}. PhoneLab provides a platform for 
people to run Android apps on their participants' smartphones 
and log data from these devices. They recruit participants by 
giving them an Android device and data plan at a low cost, in 
exchange for a commitment to use the phone as their personal 
device. However, this does not solve the security and privacy issue. 
%The participants in this case are students and faculties on campus, 
%and thus not representable of the general smart device users. 
%The research results could potentially be biased.
Furthermore, with such a usage model, researchers from different  
groups are not able to test their hypothesis at a broad scale.
%Even with participants recruited, researchers from different 
%groups cannot test their hypothesis at a world-wide scale. 
%or reuse each other's user base, experiment code, and infrastructure.
In contrast, Sensibility Testbed in this work
%introduced in this work addresses many of these concerns by 
%providing a secure, sandboxed environment 
allows \textit{any} researchers to run experiment code on ordinary people's 
devices, and \textit{any} device owners can feel free to 
participate in research initiatives without worrying about the risk 
to their personal devices.

Another challenge for researchers is the tedious work
of recruiting participants and ensuring their privacy. 
Research has shown that device owners usually do not understand the 
basics of privacy, or the implication of granting device 
permissions~\cite{camp2015respecting}. 
%Second, it is challenging for researchers to perform meaningful
%research related to end users without compromising data privacy
%and ethical merit. 
Therefore, to preserve privacy, research institutions have established 
institutional review boards (IRB) \cite{irb} and adopted formal 
procedures for approving and monitoring research involving 
human subjects. IRBs provide researchers with guidelines about 
%Due to the different goals and unforeseen risks of each
%individual project, the 
subjects, such as who can be recruited (e.g., subjects
who are over the age of 18), and 
%Researchers then need to recruit subjects for 
%each research study according to the guideline, 
how to obtain subjects' informed consent. These boards also 
regulate what data can be collected from subjects and how 
such data can be used. The goal is
to ensure the research is in compliance with ethical standards. 
However, this process is time-consuming, and the researcher needs 
to repeat it everytime he or she starts a new project. 
%For example, a researcher who uses 
%PhoneLab at the University of Buffalo cannot share the same 
%user base with Community Seismic Network~\cite{csn} at Caltech, 
%and vice versa. 
%To protect the privacy of end-user generated data, the 
%current approaches to ensuring the research 
%compliance with ethical standard is through IRB. However, 

Since the present network testbeds do not yet have a systematic 
way to protect device owners' privacy, they place 
the burden on researchers to ensure every 
experiment is compliant with their individual IRB policies.
%which can be different at each institution.
%and implement experiments with different standards. 
This requires a huge commitment 
of time and energy. Each research group has to repeat the process 
of infrastructure setup, and policy implementation for each 
experiment deployment. The resulting 
experiment code, devices, and experiment setup are 
difficult for external research groups to reuse.


% it is difficult and uncommon
%for researchers to reuse each other's user base, experiment code, 
%and infrastructure. 

In this work, we introduce a mobile testbed, 
Sensibility Testbed~\cite{sensibility, zhuang2015privacy}, a public, 
Internet-wide testbed that allows researchers to run code on ordinary
people's smartphones. This testbed addresses both of the 
aforementioned challenges.
%   
In contrast to prior testbeds, Sensibility Testbed enforces IRB policies
at end-user devices, on behalf of the researcher. The testbed's 
default policies provides a common denominator to all researchers' 
IRB policies. \yanyan{later mention: disable risky sensors by default}
The default policies are also customizable, thus serving as a template 
for researchers to parameterize their experiment. The testbed 
infrastructure then codifies researchers' IRB policies and 
implements these policies on end-user devices. Each 
policy is enforced by a \textit{blurring layer} in a secure 
sandbox. Different policies can be tailored for each experiment by loading
individual blurring layers in order, as a \textit{policy stack}. 
%Furthermore, the testbed's default set of policies provides a 
%common denominator to all researchers' IRB policies. This 
%provides the basic level of privacy protection. 
Last but not least, Sensibility Testbed allows
any device owner to participate as an anonymous volunteer,
without putting the device at risk. It can  
maximize the reuse of experiment infrastructure, thus reducing 
experimenters' recruitment and deployment burden. 
%
%there are two ways for researchers to conduct mobile 
%experiments while compliant with the ethical standard. 
%First, many mobile research projects conduct experiments 
%by either distributing an app through a marketplace, or using a 
%small number of lab participants (e.g., by distributing devices to 
%lab assistants). This usage model does not require any 
%platform support (with the exception of a marketplace), but it 
%is difficult for external research groups to reuse the experiment 
%code, the participating devices, or reproduce the experiment 
%setup. Second, research projects can be carried out on 
%smartphone testbeds, such as Phonelab~\cite{phonelab, 
%nandugudi2013phonelab} and 
%Mobilyzer~\cite{nikravesh2015mobilyzer}. 
%The problem with these testbeds is that they do not yet
%have a systematic way to protect device owners' security and 
%privacy. Various security and privacy mechanisms thus place 
%the burden on researchers, where they must ensure every 
%experiment is compliant to the IRB policies and implement 
%experiments with different standards. This constitutes
%significant overhead for an experiment deployment. 
%
%
%With the above usage models, 
%researchers from different research groups are not able to 
%test their hypothesis at a world-wide scale. As a result, 
%different research groups have to repeat effort in their individual 
%experiment deployment, infrastructure set up, policy implementation, etc. 
%or reuse each others' user and code base.
%
			
The research presented in this paper is a first step towards lowering the technical
barriers to research on personal mobile devices without lowering the
ethical standards of the research institution~\cite{zevenbergen2013ethical}. 
%while still protecting the security and privacy of end users. 
%Sensibility Testbed ensures the security of user-owned devices and the 
%privacy of user-generated data.  
%
%Sensibility Testbed is a unique research testbed in several ways.
%First, %the usage model of Sensibility Testbed is
%%unique in that 
%it maintains testbed services to manage how device 
%owners make their devices accessible to different research 
%communities anonymously, without putting
%their devices at risk. 
%%Sensibility Testbed offers technical measures
%%that allow researchers to collect data from remote mobile
%%devices without impairing the device owner's privacy by codifying 
%%experimenters' IRB policies on end devices. 
%Second, Sensibility Testbed relieves researchers from the burden of
%recruiting subjects for every single experiment, and 
%experimenters can share their code and user base by reusing
%our testbed infrastructure. The device
%owners need only give consent once, instead of 
%consenting to each project of each researcher separately.\yanyan{device owners
%can opt out an experiment: not implemented yet but shall we 
%mention this?}
%
Our work makes experiment
prototyping faster, the remote control and management of devices
easier, and running experiment code more secure. 

The contributions of this work are as follows:

\begin{enumerate}
%\item We design Sensibility Testbed, which uses security and 
%privacy techniques to minimize the technical challenges for 
%network researchers to perform meaningful research, without 
%without compromising the ethical standards outlined by the 
%research institution.

\item We design and implement Sensibility Testbed, which provides privacy protection by 
codifying experimenters' IRB policies and implementing 
these policies on end-user devices. Different policies can 
be customized, and thus the testbed can cater to a wide 
range of research initiatives.

%\item Sensibility Testbed supports generation of data access 
%policies, specified by a researcher, and implementation of 
%these policies on mobile devices to protect the privacy of 
%device owners.

\item Sensibility Testbed allows any device owner to participate 
as an anonymous volunteer, without putting the device at risk. 
It also reduces experimenters' burden, and experimenters can share 
code and user base through our testbed infrastructure.

%A consortium structure for minimizing the pain of IRB 
%approvals.  This enables participants to opt in to a large 
%array of very low risk experiments and for experimenters to 
%get a blanket approval to work with that pool of participants.  
%This eliminates the need for every experiment to get approval 
%from every participant.\yanyan{not sure about this.}

\item We implement Sensibility Testbed, evaluate its effectiveness
in protecting device owners' privacy. \yanyan{to be continued.}
\end{enumerate}

The rest of this paper is organized as follows. We present our
motivation in Section~\ref{sec-motivation}. Then in Section~\ref{sec-design}, 
we introduce the design of Sensibility Testbed. Section~\ref{sec-policy}
describes how IRB policies are codified and implemented on end-user
devices. Section~\ref{sec-eval} presents our evaluation of Sensibility 
Testbed. Section~\ref{sec-limitation} addresses the limitation of our work.
