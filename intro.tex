\section{Introduction}

Today's end-user mobile devices, such as smartphones and
tablets, have become indispensable gadgets in people's everyday
life.  As a result, the Internet architecture, end-user traffic
patterns, etc., have also evolved rapidly. These devices 
generate useful information for service providers and policy
makers to enhance network research and the services provided.
Because these devices are omnipresent, they could be of tremendous
value to the research community.
%Therefore, they have created increasing research opportunities. 
Additionally, they also provide a platform for researchers to
investigate how new applications can provide better performance.
The research potential of personal devices has sparked a number 
of recent research initiatives.
For example, there has been significant interests in the network
community to study mobile network and devices, and in the
systems community to deploy new services and test research
prototypes, such as Phonelab~\cite{phonelab, nandugudi2013phonelab}, 
Mobilyzer~\cite{nikravesh2015mobilyzer}, etc.
					
For researchers, however, there are two obstacles for research
about mobile devices. First, the privacy and security challenges
have increased dramatically over the years. The use of 
smartphones and tablets has introduced a new class of threats. 
For example, a smartphone's GPS locations,
WiFi connections, or Bluetooth pairing history can be highly
personal; a malicious party could also potentially bypass a
device's security protections and gain access to user
privileges. In particular, some seemingly benign apps can provide 
companies or governments access to ordinary 
people's daily activities~\cite{AngryBirds}. The device owners, 
however, usually do not understand the basics of privacy, or the 
implication of device permissions~\cite{camp2015respecting}.
A good system design should respect the privay of device
owners. Therefore, we not only need to ensure the security of a device
so that researcher's code cannot inadvertently damage or
maliciously hack into the device, but also protect the privacy
of device owners so that the code cannot eavesdrop on phone
conversations or infer passwords.

Second, it is challenging for researchers to perform meaningful
research related to end users without compromising data privacy
and ethical merit. Research institutions have established 
institutional review board~\cite{irb} and adopted formal 
procedures for approving and monitoring research involving 
human subjects. IRB provides researchers with guidelines about 
%Due to the different goals and unforeseen risks of each
%individual project, the 
subjects, such as who can be recruited (e.g., subjects
who are above 18), 
%Researchers then need to recruit subjects for 
%each research study according to the guideline, 
how to obtain subjects' informed consent, etc. IRB also 
regulates data collected from the subjects, e.g., what data can be collected, what can 
be done after the data has been collected, and so on. This
process is time-consuming, and the researcher needs to repeat
the process for different projects. Even with participants 
recruited according to the policy, researchers from different 
groups cannot test their hypothesis at a world-wide scale, or
reuse each other's user base, experiment code, and infrastructure.
  
Currently, there are two ways for researchers to conduct mobile 
experiments while compliant with the ethical standard. 
First, many mobile research projects conduct experiments 
by either distributing an app through a marketplace, or using a 
small number of lab participants (e.g., by distributing devices to 
lab assistants)~\cite{hao2013isleep, wang2012no, 
wang2013sensing}. This usage model does not require any 
platform support (with the exception of a marketplace), but it 
is difficult for external research groups to reuse the experiment 
code, the participating devices, or reproduce the experiment 
setup. Second, research projects can be carried out on 
smartphone testbeds, such as Phonelab~\cite{phonelab, 
nandugudi2013phonelab} and 
Mobilyzer~\cite{nikravesh2015mobilyzer}. 
The problem with these testbeds is that they do not yet 
have a systematic way to protect device owners' security and 
privacy. To lower potential risks, many of them choose to 
recruit participants from a trusted group, such as students and 
colleagues. For example, PhoneLab provides a platform for 
people to run Android apps on their participants' smartphones 
and log data from these devices. %Each participant is provided 
%with an Android device and data plan at a very low cost. 
The participants in this case are on-campus students and faculty, 
and thus not representable of the general smart device users. 
The research results could potentially 
be biased. Although PhoneLab requires experimenters to 
submit an IRB approval letter for human subjects compliance, 
the experimenters have to enforce the IRB policies in their 
experiments~\cite{nandugudi2013phonelab}. This constitutes
significant overhead for an experiment deployment. On the other 
hand, Mobilyzer~\cite{nikravesh2015mobilyzer} is distributed 
as a library that can be included in Android apps, and requires 
explicit user consent. However, there is no gurantee that an 
experiment is compliant with the researcher's IRB policies.
Moreover, with the above usage models, researchers from 
different research groups are not able to test their hypothesis 
at a world-wide scale. 
%or reuse each others' user and code base.
					
The research presented in this paper is a first step towards lowering the technical
barriers to research on personal mobile devices without lowering the
ethical standards of the research institution~\cite{zevenbergen2013ethical}, while still 
protecting the security and privacy of end users. We design and implement 
Sensibility Testbed~\cite{sensibility, zhuang2014sensibility}, a 
public, Internet-wide testbed for mobile devices that
allows researchers to run code and deploy services on ordinary
people's smartphones or tablets for research purpose. Sensibility Testbed ensures
the security of user-owned devices and the privacy of
user-generated data. Our testbed 
infrastructure codifies experimenters' IRB policies and 
implements these policies on end-user devices. Our work also 
maximizes the reuse of experiment infrastructure, thus reducing 
experimenters' recruitment and deployment burden. 

Sensibility Testbed is a unique research testbed in several ways.
First, the usage model of Sensibility Testbed is
unique in that it maintains testbed services to manage how device 
owners make their devices accessible to different research 
communities anonymously, without putting
their devices at risk. Sensibility Testbed offers technical measures
that allow researchers to collect data from remote mobile
devices without impairing the device owner's privacy by codifying 
experimenters' IRB policies on end devices. 
Sensibility Testbed also relieves researchers from the burden of
recruiting subjects for every single experiment, and 
experimenters can share their code and user base by reusing
our testbed infrastructure; the device
owners need only give consent once, instead of 
consenting to each project of each researcher separately.\yanyan{device owners
can opt out an experiment: not implemented yet but shall we 
mention this?}

Sensibility Testbed makes experiment
prototyping faster, the remote control and management of devices
easier, and running experiment code more secure. The
contributions of this work are as follows:

\begin{enumerate}
\item We design Sensibility Testbed, which uses security and 
privacy techniques to minimize the technical challenges for 
network researchers to perform meaningful research, without 
without compromising the ethical standards outlined by the 
research institution.

\item Sensibility Testbed supports generation of data access 
policies, specified by a researcher, and implementation of 
these policies on mobile devices to protect the privacy of 
device owners.

\item Sensibility Testbed reduces experimenters' burden of 
recruiting subjects; experimenters can share their code and 
user base by reusing our testbed infrastructure.

%A consortium structure for minimizing the pain of IRB 
%approvals.  This enables participants to opt in to a large 
%array of very low risk experiments and for experimenters to 
%get a blanket approval to work with that pool of participants.  
%This eliminates the need for every experiment to get approval 
%from every participant.\yanyan{not sure about this.}

\item We implement Sensibility Testbed, evaluate its effectiveness
in protecting device owners' privacy, and its performance 
\yanyan{what performance can we show?}
\end{enumerate}

The rest of this paper is organized as follows. In Section~\ref{sec-design}
we introduce the design of Sensibility Testbed.
