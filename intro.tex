\section{Introduction}

End-user mobile devices, such as smartphones and tablets, have become
indispensable gadgets in people's everyday life. %One study, conducted
%in 2015 by the Pew Research Center
Research has shown that nearly two-thirds of
Americans own a smartphone, and 19\% of them
%in that group observed that 
use the phone as their only means of staying connected~\cite{phone2015}. 
For many people, these devices have become the dominant way they
interact with others, and with the physical world.

Given the sheer numbers of these devices and the increasing
sophistication of their features,
the value of smart devices as data collection vehicles for government,
university, or corporate studies continues to grow. Since
these devices have embedded features like GPS,
accelerometers, cameras, and microphones, they can generate valuable
data for such studies as determining noise levels within an urban
neighborhood, detecting approaching earthquakes, or studying traffic
patterns at particular intersections. Accessing devices in a home
network can also help providers improve the types of services they 
offer~\cite{sundaresan2011broadband}. Testing applications on remote
devices allows developers to better understand how applications 
perform in diverse environments, enabling improvements in 
performance~\cite{ravindranath2012appinsight}. For instance, 
some platform APIs change their behavior depending 
on the battery level of the device~\cite{battery}. Without being able to remotely access battery life, these APIs can not guarantee basic function efficiency.
%Even being able to
%remotely assess how much life is left in the battery of a device can
%help platform APIs deliver better service to its
%customers~\cite{battery}.

There have been initiatives within the network
community to study mobile devices(e.g.,
Mobilyzer~\cite{nikravesh2015mobilyzer}), and in the systems community
to deploy new services and test research prototypes (e.g.,
Phonelab~\cite{phonelab, nandugudi2013phonelab}). However, personal devices
remain largely underused because of two interrelated challenges:

\begin{itemize}
\item The risk research studies pose to the privacy of \textbf{device 
owners} and the performance of applications, and 

\item The difficulties \textbf{researchers} have
securing access to devices, and using any data gathered in a
responsible and ethical manner. 
\end{itemize}
					
For device owners, privacy and security threats to mobile devices have
increased dramatically over the years as potential attackers seek
to take advantage of the rich functionality that %and user experience that
sensors\footnote{\scriptsize In this work, we broadly define sensors
as the hardware components that can record phenomena about the
physical world, such as the WiFi/cellular network, GPS location,
movement acceleration, etc.} on mobile devices can provide.
Data acquired through a smartphone's GPS,WiFi
connections, or Bluetooth pairing history can be highly personal,
exposing sensitive information, such as where a person lives or 
shops~\cite{han2012accomplice}. Even seemingly benign applications, 
such as popular online games downloaded to mobile devices, can
leak data, such as the model number of the device, or the age, gender, 
or location of its owners. \lois{the citation was deleted somewhere along the lines. Its Angry Birds, I think} Furthermore, running experiment code poses 
a risk to the operation of the device itself, through potential exposure 
to bugs and other vulnerabilities. 
%It can also seriously interfere with battery life, if the device 
%is accessed too often.

For potential testbed users, the challenges are equally formidable. 
These experimenters are under the governance of Institutional 
Review Boards (IRBs)~\cite{irb} that
review all experimental protocols involving human subjects,
and set a strict set of procedures that any researcher working under
the aegis of the institution is required to follow. These include
careful control over the collection and storage of data to ensure the 
privacy of subjects is preserved. It falls on the researcher to enforce 
these policies, and the process must be repeated every time he or 
she starts a new project. The experimenter must also recruit device 
owners willing to volunteer their devices for testing, a process that 
is time-consuming. Moreover, experimental setup and results cannot 
easily be shared with other researchers. As a result, each research 
group has to repeat the process of infrastructure setup, and policy 
implementation for each experimental deployment.

To address some of these concerns, experimental testbeds such as
PhoneLab, have
been established to provide a platform for running apps on
smartphones. PhoneLab recruits participants by giving them free
smartphones and reduced data plans in exchange for their commitment to
use the phones as their personal devices. 
%PhoneLab then runs Android apps on the devices and collects data. 
Other testbeds deal with both
the recruitment and privacy issues by choosing to select participants
from an internal group, such as faculty working with their students
and colleagues~\cite{hao2013isleep, wang2012no, wang2013sensing}. However,
such controlled candidate selections fall short for a few reasons.
First, it does not relieve the burden on the researchers to ensure the privacy of the
device owner and to enforce IRB policies, since the present
network testbeds do not yet have a systematic way to provide these protections. As a result, there is limited protection for the device 
owners. Research has shown that most people usually do not understand 
the basics of privacy, or the implication of granting device
permissions~\cite{camp2015respecting}. Therefore, if the testbed does 
not safeguard the security of devices, no matter what 
advantages participants may receive, their devices are still at risk.

In this work, we introduce Sensibility Testbed~\cite{sensibility,
zhuang2015privacy}, an Internet-wide mobile testbed that 
%represents an important first step towards 
lowers the technical barriers to research on personal mobile
devices without lowering the ethical standards of the research
institution~\cite{zevenbergen2013ethical}.
The new testbed makes experiment prototyping faster, the remote
control and management of devices easier, and the running of
experiment code more secure in a number of ways. First, 
it provides better protection against invasion of privacy by carefully controlling
access to device sensors. The testbed employs a stringent set of
policies as to which sensors can be accessed, and these
policies are customizable to each researchers' IRB policies. Therefore,  
this serves as a template for researchers to parameterize their 
experiment. The testbed infrastructure automatically implements
the IRB policies on end-user devices, through the use of \textit{blurring 
layers} in a secure sandbox. Each blurring layer mediates the access to 
a sensor by limiting the precision of data generated by it, and 
regulating the frequency that the sensor can be accessed. In addition, 
the testbed's secure sandbox provides both security and performance 
isolation, and ensures experiment code can not harm the devices of 
volunteers. Due to these privacy and security mechanisms, 
the enrollment process for volunteer device owners is as
simple as a one-time download and install of an app. The testbed thus
builds and maintains a pool of willing volunteers for researchers to
choose from, eliminating the time-consuming process of recruitment.


The contributions of this work are as follows:

\begin{enumerate}
\item We identify the issues that have prevented successful implementation
of experiments on remote user devices, including potential damage to
devices from experiment code, the risk of privacy invasion, and the
administrative challenges faced potential researchers. \yanyan{I fell this may not
be significant enough as a contribution.}

\item We introduce Sensibility Testbed as a flexible and secure platform for
conducting experiments on Android devices that addresses many of the
concerns above, including automatically obtaining and enforcing 
institution-mandated IRB policies.

\item We describe the unique features of Sensibility Testbed, which include 
controlled sensor access through the development of blurring layers.

\item We evaluate Sensibility Testbed's effectiveness in
protecting device owners' privacy and find that nearly 86\% of
security and privacy issues present in previous projects using 
mobile devices were addressed by the new system.
\end{enumerate}

The rest of this paper is organized as follows. First, in Section~\ref{sec-motivation} we
present background information about several key concepts that are
critical to understanding how Sensibility Testbed works and how its use can benefit both researchers and device owners. Section~\ref{sec-design} offers an overview of the design principles that guided  development of the testbed, including a description of its key components and a simplified look at the operation of the program. The architecture of the Sensibility Testbed is reviewed in Section~\ref{sec-policy}. Section~\ref{sec-eval} presents a detailed look at the implementation of Sensibility Testbed , while Section offers a detailed walkthough of the program in operation. Section Seven provides experimental results to prove the
viability of Sensibility Testbed in enforcing privacy policies, while
Section~\ref{sec-limitation} examines challenges and current limitations for
implementation of the testbed. In Section~\ref{sec-related} we review related work
in protecting the privacy of data on mobile devices, and we share some
concluding thoughts in Section~\ref{sec-conclude}.

