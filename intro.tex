\section{Introduction}

End-user mobile devices, such as smartphones and tablets, have become
indispensable gadgets in people's everyday life. One study, conducted
in 2015 by the Pew Research Center~\cite{phone2015}, found that nearly two-thirds of
all Americans own a smartphone, and 19\% of the individuals in that
group observed that the phone is their only means of accessing the
Internet, or for staying connected to the world around them. In short,
for many people, these devices have become the dominant way to
interact with others and with the physical world.

Given both the sheer numbers of these devices and the increasing
sophistication of the recording and navigating features they possess,
the value of smart devices as data collection vehicles for government,
university, or corporate studies continues to grow as well. Since
these smart phones and tablets employ features such as GPS,
accelerometers, cameras, and microphones, they can generate valuable
data for such studies as determining noise levels within an urban
neighborhood, detecting approaching earthquakes, or studying traffic
patterns at particular intersections. Accessing devices in a home
network can also help providers improve services offered to their
customers~\cite{sundaresan2011broadband}, and the quality of
applications, by letting developers better understand how they perform
in diverse external or internal
conditions~\cite{ravindranath2012appinsight}. Even being able to
remotely assess how much life is left in the battery of a device can
help platform APIs deliver better service to its
customers~\cite{battery}.

But, while there have been a few high-profile initiatives within the network
community to study mobile devices(e.g.,
Mobilyzer~\cite{nikravesh2015mobilyzer}), and in the systems community
to deploy new services and test research prototypes (e.g.,
Phonelab~\cite{phonelab, nandugudi2013phonelab}), personal devices
remain largely underused because of two interrelated problems. The
first is the very real risk such studies pose to the privacy of the
device owner, and the other is the difficulties researchers have
securing access to devices, and using any data gathered in a
responsible and ethical manner. These issues can be summarized as follows:
					
For device owners, privacy and security threats to mobile devices have
increased dramatically over the years as potential attackers also seek
to take advantage of the rich functionality and user experience that
sensors\footnote{\scriptsize In this work, we broadly define sensors
as the hardware components that can record phenomena about the
physical world, such as the WiFi/cellular network, GPS location,
movement acceleration, etc.} on smartphones and tablets can provide.

\begin{enumerate}
\item Data acquired through a smartphone's GPS locations,WiFi
connections, or Bluetooth pairing history can be highly personal,
exposing sensitive information, such as where a person lives or shops\cite{han2012accomplice}.

\item  Even seemingly benign applications, such as popular online games downloaded to mobile devices, can
 "leak" data, such as the model number and screen size of the phone, or worse the age, gender, or location of the user. 

\item Running experiment code poses a real risk to the operation of the device itself, through potential exposure to bugs or other 
vulnerabilities. It can also seriously interfere with battery life, if the device is accessed too often.
\end{enumerate}

For potential testbed user, the challenges are equally formidable:
\begin{itemize}
\item Potential experimenters are under the governance of Institutional Review Boards (IRBs) that
review all experimental protocols involving human subjects in advance,
and set a strict set of procedures that any researcher working under
the aegis of the institution is required to abide by. These includes
careful control over the collection and storage of data to ensure the privacy of subjects is preserved.

\item  It falls on the researcher to enforce these policies, and the
process must be repeated every time he or she starts a new project.

\item The experimenter must also recruit owners willing to volunteer their devices for testing, a process that can be very time-consuming.

\item Experimental set-ups and results can not easily be shared with other researchers, so each research group has to repeat the process
of infrastructure setup, and policy implementation for
each experimental deployment.
\end{enumerate}

To address some of these concerns, experimental testbeds such as
PhoneLab, have
been established to provide a platform for running apps on
smartphones. PhoneLab recruits participants by giving them free
smartphones and reduced data plans in exchange for their commitment to
use the phones as their personal devices. PhoneLab then runs Android
apps on the devices and collects data. Other testbeds deal with both
the recruitment and privacy issues by choosing to select participants
from an internal group, such as faculty working with their students
and colleagues~\cite{hao2013isleep, wang2012no, wang2013sensing}. But,
such controlled candidate selections fall short for a few reasons.
First, it does not relieve the burden on the researcher to ensure the privacy of the
device owner and to enforce any IRB policies, since the present
network testbeds do not yet have a systematic way to protect device
owners' privacy. Plus,
there is limited protection for the security of device owners. Since
research has shown that device owners usually do not understand the
basics of privacy, or the implication of granting device
permissions~\cite{camp2015respecting}, if the testbed is not taking on
the responsibility of safeguarding the security of the devices, then
no matter what advantages participants may receive, they are still
taking a risk if they volunteer.

In this work, we introduce Sensibility Testbed~\cite{sensibility,
zhuang2015privacy}, a public,
Internet-wide mobile testbed that represents an important first step
towards lowering the technical barriers to research on personal mobile
devices without lowering the
ethical standards of the research
institution~\cite{zevenbergen2013ethical}.
The new testbed makes experiment prototyping faster, the remote
control and management of devices easier, and the running of
experiment code more secure in a number of ways.

\begin{enumerate}
\item  it provides better protection against invasion of privacy by carefully controlling
access to device sensors. The testbed employs a stringent set of
default policies as to which sensors can be accessed, and these
policies are customizable, thus serving as a template for researchers
to parameterize their experiment

\item   the testbed infrastructure can codify each researchers' IRB policies, and can
directly and automatically implement these policies on end-user
devices. 

\item  Through the use of \textit{blurring layers} in a secure
sandbox, Sensibility Testbed can limit documentation of his/her exact
location to the nearest city, and can regulate the amount of times
over the course of the day that the sensors can be accessed.

\item The testbed employs a Repy Sandbox, which provides both security and performance isolation,
through a flexible system call interposition that ensures experiment code can not harm the
devices of volunteers, or generate data that could be invasive to
their privacy.

\item  The enrollment process for volunteer device owners is as
simple as a one-time download and install of an app. The testbed thus
builds and maintains a pool of willing volunteers for researchers to
choose from, eliminating the time-consuming process of recruitment.
\end{enumerate}

The contributions of this work are as follows:

\begin{enumerate}
\item We identify the issues that have prevented successful implementation
of experiments on remote user devices, including potential damage to
devices from experiment code, the risk of privacy invasion, and the
administrative challenges faced potential researchers.

\item We introduce Sensibility Testbed as a flexible and secure platform for
conducting experiments on Android devices that addresses many of the
concerns above, including automatically obtaining and enforcing institution-mandated IRB policies.

\item We describe the unique features of Sensibility Testbed, which includes controlled sensor access through the
development of blurring layers.

\item We evaluate the Sensibility Testbed's effectiveness in
protecting device owners' privacy and found that nearly 86\% of
security and privacy issues present in previous tests on the use of
mobile devices were addressed by the new system.
\end{enumerate}

The rest of this paper is organized as follows. First, in Section~\ref{sec-motivation} we
present background information about several key concepts that are
critical to understanding how Sensibility Testbed works and what makes
it unique. Section~\ref{sec-design} offers a detailed breakdown about the design of
the testbed, including a description of its key components and how its
use is implemented by both a research and a device owner. The specific
features of Sensibility Testbed that ensures that volunteers can
participate without damage to their phone or their data are reviewed
in Section~\ref{sec-policy}. Section~\ref{sec-eval} presents results of tests to prove the
viability of Sensibility Testbed in enforcing privacy policies, while
Section~\ref{sec-limitation} examines challenges and current limitations for
implementation of the testbed. In Section~\ref{sec-related} we review related work
in protecting the privacy of data on mobile devices, and we share some
concluding thoughts in Section~\ref{sec-conclude}.

