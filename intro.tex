\section{Introduction}

End-user mobile devices, such as smartphones and tablets, have become
indispensable gadgets in people's everyday life. %One study, conducted
%in 2015 by the Pew Research Center
Research has shown that nearly two-thirds of
Americans own a smartphone, and 19\% of them
%in that group observed that 
use the phone as their only means of staying connected~\cite{phone2015}. 
For many people, these devices have become the dominant way they
interact with others, and with the physical world.

Given the sheer number of these devices and the increasing
sophistication of their features,
the value of smart devices as data collection vehicles for government,
university, or corporate studies continues to grow. Since
these devices have embedded GPS,
accelerometers, cameras, and microphones, they can generate valuable
data for such studies as determining noise levels within an urban
neighborhood~\cite{kardous2014evaluation}, detecting approaching 
earthquakes~\cite{faulkner2011next}, or studying traffic
patterns at particular intersections~\cite{zhuang2011time}. Accessing 
devices in a home network can help providers improve the service 
quality~\cite{sundaresan2011broadband}. Testing applications on remote
devices allows developers to better understand how applications 
perform in diverse environments, enabling improvements in 
performance~\cite{ravindranath2012appinsight}. 
%For instance, some platform APIs change their behavior depending 
%on the battery level of the device~\cite{battery}. 
%Without remote access battery life, these APIs can not guarantee basic function efficiency.
%Even being able to
%remotely assess how much life is left in the battery of a device can
%help platform APIs deliver better service to its
%customers~\cite{battery}.
%
As a result, there have been initiatives within the network
community to study mobile devices (e.g.,
Mobilyzer~\cite{nikravesh2015mobilyzer}), and in the systems community
to deploy new services and test research prototypes (e.g.,
Phonelab~\cite{phonelab, nandugudi2013phonelab}). However, personal devices
remain largely underused because of two interrelated challenges:

\begin{itemize}\setlength\itemsep{0em}
\item The risk research studies pose to the privacy of \textbf{device 
owners}, and 

\item The difficulties \textbf{researchers} have
%\cappos{Do you mean obtaining access due to IRB constraints?  What does 
%"securing" mean here?}
obtaining access to devices due to the restrictions set by Institutional Review 
Boards, and using data gathered in a responsible and ethical manner. 
\end{itemize}
					
For device owners, privacy and security threats to mobile devices have
increased dramatically over the years as potential attackers seek
to take advantage of the rich functionality that %and user experience that
sensors\footnote{\scriptsize In this work, we broadly define sensors
as the hardware components that can record phenomena about the
physical world, such as the WiFi/cellular network, GPS location,
movement acceleration, etc.} on mobile devices can provide.
Data acquired through a smartphone's GPS,WiFi
connections, or Bluetooth pairing history can be highly personal,
exposing sensitive information, such as where a person lives or 
shops~\cite{han2012accomplice}. Seemingly benign applications, 
such as popular online games downloaded to mobile devices, can
leak data, such as the model number of the device, or the age, gender, 
or location of its owners~\cite{AngryBirds}. Furthermore, running experiment code poses 
a risk to the operation of the device itself through potential exposure 
to bugs and other vulnerabilities. 
%It can also seriously interfere with battery life, if the device 
%is accessed too often.

For researchers, the challenges are equally formidable. 
Researchers' experimenters are under the governance of Institutional 
Review Boards (IRBs)~\cite{irb} that
review all experimental protocols involving human subjects,
and set a strict set of procedures that any researcher working under
the aegis of the institution is required to follow. These include
careful control over the collection and storage of data to ensure the 
privacy of subjects is preserved. It falls on the researcher to enforce 
these policies, and the process must be repeated every time he or 
she starts a new project. The experimenter must also recruit device 
owners willing to volunteer their devices for testing, a process that 
is time-consuming. Moreover, experimental setup and results cannot 
easily be shared with other researchers. 
%As a result, each research group has to repeat the process of 
%infrastructure setup, and policy implementation \cappos{what 
%is this?} for each experimental deployment.

Due to these concerns, experimental testbeds either do not use data 
from real participants (using archived traces like 
in~\cite{kapadia2008anonysense}) or use data collected 
from a tightly controlled, non-representative subset of participants.
%such as PhoneLab, have been established to provide a platform for 
%running apps on smartphones. 
%PhoneLab recruits participants by giving them free
%smartphones and reduced data plans in exchange for their commitment to
%use the phones as their personal devices. 
%PhoneLab then runs Android apps on the devices and collects data. 
For example, some testbeds deal with both
the recruitment and privacy issues by choosing to select participants
from an internal group, such as faculty working with their students
and colleagues~\cite{hao2013isleep, wang2012no, wang2013sensing}. 
Other testbeds require researchers obtain IRB approval prior to using the
testbed, but provide no technical measures for IRB policy 
enforcement~\cite{phonelab, nandugudi2013phonelab}. 
These mechanisms still fall short, as they do not 
relieve the burden on the researchers to ensure the privacy of the
device owner and to enforce IRB policies. 
%Since the present network testbeds do not yet have a systematic way 
%to provide these protections, there is limited protection for the device 
%owners. 
%Research has shown that most people usually do not understand 
%the basics of privacy, or the implication of granting device
%permissions~\cite{camp2015respecting}. Therefore, if the testbed does 
%not safeguard the security of devices, no matter what 
%advantages participants may receive, their devices are still at risk.
%\cappos{I miss the point of this paragraph in the rest of the narative...}

In this work, we introduce Sensibility Testbed~\cite{sensibility,
zhuang2015privacy}, an Internet-wide mobile testbed that 
%represents an important first step towards 
lowers the technical barriers to research on personal mobile
devices without lowering the ethical standards of the research
institution~\cite{zevenbergen2013ethical}.  
%\cappos{I would instead emphasize 
%that it streamlines IRB since it acts as a middle man for experimenters and 
%participants.  The paragraph here seems to wander to me (e.g., where do we
%discuss ``making prototyping faster'' again?), but eventually comes
%to this point anyways.}
The new testbed streamlines the IRB process for researchers as it acts 
as an intermediate between researchers and device owners. 
%makes experiment prototyping faster, the remote
%control and management of devices easier, and the running of
%experiment code more secure in a number of ways. 
Researchers provide their policy specification to the testbed, and 
the testbed recruits participants and implements the policies on researchers' 
behalf. This model 
makes experiment prototyping faster and more secure in a number of ways. 

First, \sysname provides better protection against invasion of privacy by carefully controlling
access to device sensors. The testbed employs a stringent set of
policies as to which sensors can be accessed, and these
policies are customizable to each researchers' IRB policies. 
%This serves as a template for researchers to parameterize their 
%experiment. 
Second, the testbed infrastructure automatically implements
the IRB policies on end-user devices, through the use of \textit{blurring 
layers} in a secure sandbox. Each blurring layer mediates the access to 
a sensor by limiting the precision of data generated by it, by 
regulating the frequency that the sensor can be accessed, or both. Policy thus
can be easily enforced at the end-user devices by the testbed 
infrastructure. In addition, 
the testbed's secure sandbox provides both security and performance 
isolation, and ensures experiment code cannot harm the devices of 
volunteers. Due to these privacy and security mechanisms, 
the enrollment process for volunteer device owners is as
simple as a one-time download and install of an app. The testbed thus
builds and maintains a pool of willing participants for researchers to
choose from, eliminating the time-consuming process of recruitment.


The contributions of this work are as follows:

\begin{itemize}\setlength\itemsep{0em}
%\item We identify the issues that have prevented successful implementation
%of experiments on remote user devices, including potential damage to
%devices from experiment code, the risk of privacy invasion, and the
%administrative challenges faced potential researchers. 

\item We introduce Sensibility Testbed as a flexible and secure platform for
conducting experiments on Android devices that addresses the
concerns above, including automatically obtaining and enforcing 
institution-mandated IRB policies. \sysname identifies potential 
risk of privacy invasion and alleviates administrative 
challenges faced by potential researchers.

\item We describe the unique features of Sensibility Testbed, which include 
controlled sensor access through the development of blurring layers that are
highly customizable.

\item We evaluate Sensibility Testbed's effectiveness in
protecting device owners' privacy and find that our default 
policies would address most of the 
security and privacy issues present in the literature in 
previous projects using mobile devices.
\jill{i think the contributions should be: 
1. ST streamlines the process of IRB policy enforcement, which makes experiments safer for device owners and easier for researchers who no longer have to recruit mobile device participants 
2. ST allows researchers to run multiple experiments using ST without having to obtain an IRB and rewrite code enforcing the policies of the IRB for each experiment.
3. controlled sensor access through blurring layers
4. perhaps something here about default policies for protecting the privacy of device owners as an extra precaution
5. st allows researchers to share device subjects with security isolation between experiements}
\end{itemize}

The rest of this paper is organized as follows. In Section~\ref{sec-motivation} we
present background information about several key concepts of \sysname. 
%that are critical to understanding how Sensibility Testbed works and how its use can benefit both researchers and device owners. 
Section~\ref{sec-overview} offers an overview of the basic approval and 
deployment procedure for an experiment, and introduces \sysname's 
default sensor access policies.
%including a description of its key components and a simplified look at 
The architecture of the Sensibility Testbed 
is reviewed in Section~\ref{sec-design}, and Section~\ref{sec-policy} 
presents the detailed implementation of the architecture. 
%while Section offers a detailed walkthough of the program in operation. 
Section~\ref{sec-eval} provides experimental results to prove the
viability of Sensibility Testbed in enforcing privacy policies, while
Section~\ref{sec-limitation} examines challenges and current limitations. 
In Section~\ref{sec-related} we review related work,
%in protecting the privacy of data on mobile devices, 
and we share our concluding thoughts in Section~\ref{sec-conclude}.

